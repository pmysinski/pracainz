\chapter{Wprowadzenie}

Dokumenty, listy, protokoły to nierozłączna część każdego procesu rekrutacji na uczelni wyższej. Ręczne tworzenie takich dokumentów może nastręczać wielu problemów. Przede wszystkim głównym problemem jest możliwość wystąpienia błędów człowieka. W nawet najlepiej zorganizowanej placówce szkolnej może wkraść się błąd, który spowoduje iż na przykład kandydat zostanie przypadkowo odrzucony lub przyjęty na studia. Usuwanie takich błędów może być czasami nie możliwe, dlatego ręczne tworzenie takich dokumentów zajmuje wiele czasu, aby mieć pewność, by nie popełnić błędu. Przy czym istotnym jest, że rekrutacja powinna przebiegać szybko ze względu na fakt, iż kandydaci potrzebują znać decyzję o przyjęciu w miarę szybko aby móc w razie czego zgłosić się do innych szkół w tym samym okresie. Dodać jeszcze należy, że podczas nieefektywnie zorganizowanej rekrutacji, w procesie tym udział muszą wziąć pracownicy dydaktyczni, którzy muszą poświęcać swój czas na okres rekrutacji.


\section{ Zagadnienie generowania raportów}

Z rozwiązaniem tego problemu przychodzi informatyzacja procesu rekrutacji. Pozwala ona na eliminacje praktycznie wszystkich błędów człowieka poprzez automatyczne korekty danych oraz auto uzupełnianie dokumentów danymi. Przyspiesza proces rekrutacji poprzez ułatwienie go dla osób prowadzących go oraz fakt iż dokumenty generowane są w bardzo krótkim czasie. 
\par
W procesie generowania raportów zawsze bierze udział pewien rodzaj bazy danych, z której pobierane są informacje. Dane są poddawane selekcji a następnie wklejane odpowiednie miejsce we wcześniej przygotowany szablon danego raportu. W ten sposób powstaje dokument wygenerowany z dynamicznych danych, zapisany w różnych formach. Najpopularniejszą formą zapisu jest \emph{Portable Document Format} w skrócie \emph{PDF}. Dokument wygenerowany do tego formatu pozwala na natychmiastowe przejrzenie zawartości oraz możliwość wydruku.


\section{ Dotychczasowy proces generowania raportów}

Po przeanalizowaniu obecnego systemu można stwierdzić, iż jest on mało wydajny i wymaga poprawy. Do generowanie dokumentów używana jest funkcja programu \emph{IBExpert}, \emph{Report Manager}. Stworzenie szablonu raportu w tym podprogramie jest procesem żmudnym oraz skomplikowanym dla osoby zajmującej się administracją bez wiedzy informatycznej. Dodatkowo należy dodać, że nie posiada on wystarczającej funkcjonalności, przez co proces generowania raportów nie przebiega w pełni automatycznie i wymaga pełnego nadzoru osoby wdrożonej w ten system.  Dobrym przykładem na pokazanie niedoskonałości jest fakt, iż każdy raport musi być generowany oddzielnie dla każdego kierunku studiów, stopnia czy też formy. Wytwarza to problemy związane z powstawaniem błędów czy też segregacją raportów.

\section{Cel i zakres pracy}

Celem pracy, czyli informatyzacji procesu rekrutacji, jest zapewnienie pracownikom administracji jak i również osobom upoważnionym, dostępu do narzędzia użytecznego, bezpiecznego oraz prostego w stosowaniu, które w znacznym stopniu przyspieszy ten proces oraz zapewni minimalizację błędów, które mogę powstać podczas tworzenia dokumentów potrzebnych przy rekrutacji. 
\par
Osiągnięcie postawionego celu, wymaga wykonania następujących zadań:
\begin{enumerate}
\item Wybranie systemu generowania dokumentów oraz języka do implementacji programu.
\item Utworzenie szablonów dokumentów w procesie rekrutacji odpowiadających tym, które są obecnie używane.
\item Zaimplementowanie programu, który będzie stanowił interfejs dla użytkownika oraz wykonywał algorytm uzupełniania szablonów raportów.
\item Przeprowadzenie testu działania systemu na wyczerpującej potrzeby liczbie kandydatów.
\item Przygotowanie dokumentacji oraz instrukcji obsługi systemu.
\end{enumerate}
