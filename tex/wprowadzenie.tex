\chapter{Wprowadzenie}

Listy, protokoły, decyzje oraz różnego typu dokumenty towarzyszące to nierozłączna część każdego procesu rekrutacji kandydatów na studia w uczelni wyższej. Sporządzenie wymienionych typów raportów, bez specjalizowanego oprogramowania, może prowadzić do powstania wielu problemów i związanych z nimi kosztów.  Trudno oczekiwać, aby uczelnia w momencie zamawiania takiego oprogramowania była w stanie przewidzieć jakiego rodzaju i w jakiej postaci należy sporządzać analogową dokumentację procesu rekrutacyjnego. Stąd w większości uczelni, mimo obecności dedykowanego oprogramowania rekrutacyjnego, zmusza się powołane komisje rekrutacyjne do przygotowywania wielu dokumentów w sposób manualny, z wykorzystaniem podstawowego oprogramowania do składu tekstu. Takie postępowanie narażone jest na wiele pomyłek człowieka, przy ogromie zestawianych, w różnych formach, tych samych danych.  Dodać jeszcze należy, że w skład komisji rekrutacyjnych, co roku wchodzą różni pracownicy dydaktyczni uczelni (tak jest m.in. w przedmiotowej PWSZ w Nowym Sączu), którzy od początku muszą zorganizować swój warsztat pracy w sposób nieefektywny, ponieważ komisje byłe i obecne nie wymieniają między sobą żadnych informacji naprawczych poza jedną, że praca w komisji rekrutacyjnej to prawdziwa, nieoceniona udręka.


Z rozwiązaniem przedstawionego problemu może przyjść  informatyzacja tej części procesu rekrutacyjnego, która podlega corocznym zmianom. Pozwala ona w sposób niezależny od firmy dostarczającej oprogramowanie rekrutacyjne\footnote{Musi zostać spełniony warunek, że oprogramowanie dostarczone jest wraz z bazą danych zarządzaną przez RDBMS, który umożliwia wydobyć w sposób autoryzowany określone dane. Warunek ten spełniony jest w PWSZ w Nowym Sączu, ponieważ można wydobyć określone dane za pomocą RDBMS, którym jest Firebird.}, przygotować w placówce uczelni zestaw szablonowych dokumentów rekrutacyjnych i połączyć je z danymi z systemu bazodanowego. W efekcie otrzymuje się wielostronicowe, uporządkowane zgodnie z przyjętą metodologią przekazywania raportów, pliki pdf, gotowe do wydruku na specjalizowanych urządzeniach drukująco\dywiz składająco\dywiz  kopertujących.


\section{Cel i zakres pracy}
Celem pracy jest opracowanie informatyczne systemu, który zapewni komisjom rekrutacyjnym możliwość przygotowania drukowalnych wersji raportów  rekrutacyjnych, których brak w specjalizowanym systemie uczelni. 

Dodatkowy celem jest eliminacja sytuacji, która ma miejsce w przedmiotowej uczelni, że  firma ignoruje zgłaszane uwagi o utrudnianiu, a nie wspomaganiu przez dostarczane oprogramowanie pracy komisji rekrutacyjnych, przez ogromny nakład dodatkowej pracy związanej z wykorzystaniem ich aplikacji do rekrutacji. Opracowany system ma pokazać\footnote{Opracowany system już pokazał, że całość obsługi procedur kończących rekrutację \{obejmujących: (a) kontrolę zgodności danych elektronicznych i analogowych, (b) wygenerowanie i wydrukowanie wszystkich raportów, (c) kopertowanie podpisanych decyzji do wysyłki i oklejenie kopert potwierdzeniem otrzymania decyzji, (d) umieszczanie decyzji w teczkach osobowych kandydatów (f) przekazanie całości zgromadzony akt osobowych kandydatów do sekretariatu,  (g) wywieszenie list i przekazanie formy elektronicznej do publikacji, (h) przekazanie do wysyłki decyzji z \emph{(c)}\} na liczbie 400 kandydatów w Instytucie Technicznym zajęło 2h a w Instytucie Ekonomicznym 20h -- 2 dni robocze, przy istnieniu dużej niepewności, czy się gdzieś nie doszło do pomyłki.}, że obecna procedura jest zaprojektowana nieudolnie i wymaga całkowitej naprawy. Bo nie każda aplikacja komputerowa usprawnia pracę.

Zakres prac obejmuje:
\begin{enumerate}
\item Wybór systemu do przygotowania szablonów raportów, wraz z systemem ich wizualnej prezentacji.
\item Wybór środowiska do utworzenia przenośnej wersji programu, który umożliwi wydobycie danych z baz uczeni i połączenie z szablonami w celu wygenerowania drukowanych wersji raportów w formacie pdf.
\item Utworzenie, aktualnych, szablonów dokumentów rekrutacyjnych w PWSZ w Nowym Sączu.
\item Zaimplementowanie programu oraz jego opis dokumentacyjny, umożliwiający w przyszłości wprowadzenie nowej funkcjonalności, jeżeli takie się pojawią.
\item Wygenerowanie danych testowych, symulujące rzeczywiste dane z bazy uczelnianej.
\item Przeprowadzenie testu działania systemu na wygenerowanych danych.
\item Przygotowanie dokumentacji oraz instrukcji obsługi systemu.
\item Obserwacja wykorzystania opracowanego systemu w rzeczywistym procesie rekrutacji w roku akademickim 2014, 2015, 2016 systemu w dedykowanej jednostce.
\end{enumerate}
