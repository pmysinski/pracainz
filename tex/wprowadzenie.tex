\chapter{Wprowadzenie}

Listy, protokoły, decyzje oraz różnego typu dokumenty towarzyszące to nierozłączna część każdego procesu rekrutacji kandydatów na studia w uczelni wyższej. Sporządzenie wymienionych typów raportów, bez specjalizowanego oprogramowania, może prowadzić do powstania wielu problemów i związanych z nimi kosztów.  Trudno oczekiwać, aby uczelnia w momencie zamawiania takiego oprogramowania była w stanie przewidzieć jakiego rodzaju i w jakiej formie należy sporządzać w formie analogowej dokumentację procesu rekrutacyjnego. Stąd w większości uczelni, mimo obecności dedykowanego oprogramowania rekrutacyjnego, zmusza się powołane komisje rekrutacyjne do przygotowywania wielu dokumentów w sposób manualny, z wykorzystaniem podstawowego oprogramowania do składu tekstu. Takie postępowanie narażone jest na wiele pomyłek człowieka, przy ogromie zestawianych, w różnych formach, tych samych danych.  Dodać jeszcze należy, że w skład komisji rekrutacyjnych, co roku wchodzą różni pracownicy dydaktyczni uczelni (tak jest m.in. w przedmiotowej PWSZ w Nowym Sączu), którzy od początku muszą zorganizować swój warsztat pracy w sposób nieefektywny, ponieważ komisje byłe i obecne nie wymieniają między sobą żadnych informacji naprawczych poza jedną, że praca w komisji rekrutacyjnej to prawdziwa, nieoceniona udręka.


Z rozwiązaniem przedstawionego problemu może przyjść  informatyzacja tej części procesu rekrutacyjnego, która podlega corocznym zmianom. Pozwala ona w sposób niezależny od firmy dostarczającej oprogramowanie rekrutacyjne (oprogramowanie dostarczone jest wraz z bazą danych, z której można wydobyć w sposób autoryzowany określone dane), przygotować w placówce uczelni zestaw szablonowych dokumentów i połączyć je z danymi z systemu bazodanowego. W efekcie otrzymuje się wielostronicowe, uporządkowane zgodnie z przyjętą metodologią przekazywania raportów, pliki pdf, gotowe do wydruku na specjalizowanych urządzeniach drukująco\dywiz składająco\dywiz  kopertujących.


\section{Cel i zakres pracy}
Celem pracy jest opracowanie informatyczne systemu, który zapewni komisjom rekrutacyjnym możliwość przygotowania, drukowalnych wersji raportów  rekrutacyjnych, których brak w systemie rekrutacyjnym uczelni, lub sposób ich organizacji jest z grupy pseudoinformatyczno\dywiz usprawniających pracę.

Osiągnięcie postawionego celu, wymaga wykonania następujących zadań:
\begin{enumerate}
\item Wybór systemu do przygotowania szablonów raportów, wraz z systemem ich wizualnej prezentacji.
\item Wybór środowiska do utworzenia przenośnej wersji programu, który umożliwi wydobycie danych z baz uczeni i połączenie z szablonami w celu wygenerowania drukowanych wersji raportów w formacie pdf.
\item Utworzenie, aktualnych, szablonów dokumentów rekrutacyjnych w PWSZ w Nowym Sączu.
\item Zaimplementowanie programu oraz jego opis dokumentacyjny, umożliwiający w przyszłości wprowadzenie nowej funkcjonalności, jeżeli takie się pojawią.
\item Wygenerowanie danych testowych, symulujące rzeczywiste dane z bazy uczelnianej.
\item Przeprowadzenie testu działania systemu na wygenerowanych danych.
\item Przygotowanie dokumentacji oraz instrukcji obsługi systemu.
\item Obserwacja wykorzystania opracowanego systemu w rzeczywistym procesie rekrutacji w roku akademickim 2014, 2015, 2016 systemu w dedykowanej jednostce.
\end{enumerate}
