\chapter{Wprowadzenie}

Dokumenty, listy, protokoły to nierozłączna część każdego procesu rekrutacji na uczelni wyższej. Ręczne tworzenie takich dokumentów może nastręczać wielu problemów. Przede wszystkim głównym problemem jest możliwość wystąpienia błędów człowieka. W nawet najlepiej zorganizowanej placówce szkolnej może wkraść się błąd, który spowoduje iż na przykład kandydat zostanie przypadkowo odrzucony lub przyjęty na studia. Odkręcenie takich błędów może być czasami nie możliwe, dlatego też do przy ręczne tworzenie takich dokumentów zajmuje wiele czasu aby mieć pewność aby nie popełnić błędu. Przy czym istotnym jest, że rekrutacja powinna przebiegać szybko ze względu na fakt, iż kandydaci potrzebują znać decyzję o przyjęciu w miarę szybko aby móc w razie czego zgłosić się do innych szkół w tym samym okresie. Dodać jeszcze należy, że podczas nieefektywnie zorganizowanej rekrutacji, w procesie tym udział muszą wziąć pracownicy dydaktyczni, którzy muszą poświęcać swój czas na okres rekrutacji.


\section{ Zagadnienie generowania raportów}

Z rozwiązaniem tego problemu przychodzi informatyzacja procesu rekrutacji. Pozwala ona na eliminacje praktycznie wszystkich błędów człowieka poprzez automatyczne korekty danych oraz auto uzupełnianie dokumentów danymi. Przyspiesza proces rekrutacji poprzez ułatwienie go dla osób prowadzących go oraz fakt iż dokumenty generowane są w bardzo krótkim czasie.


\section{ Dotychczasowy proces generowania raportów}

\section{Cel i zakres pracy}

