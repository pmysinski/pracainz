\chapter{Projekt programu i implementacja}

Ten rozdział przedstawia dokładny proces implementacji programu \emph{DBLatexRaport,}  który ma na celu spełnienia wymagań oraz celów zawartych w rozdziale 1.  Osoba po przeanalizowaniu poniższego materiału, będzie w stanie w przyszłości ulepszyć o nowe funkcję istniejący już program lub stworzyć nowy, podobny program w innym języku programowania, który będzie w stanie obsłużyć istniejące już szablony. 

\section{Algorytm działania systemu raportowania}

W poprzednim rozdziale przedstawiony został proces tworzenia samych szablonów, natomiast w tej sekcji skupiona uwaga zostanie na tym jak zaprojektować system uzupełniania tych szablonów. Między innymi właśnie o tym jak umieszczać informacje w szablonie na temat selekcjonowania danych czy też jak wygenerować odpowiednią strukturę danych.

\subsection{Algorytm parsowania}
Aby szablon raportu został uzupełniony o potrzebne dane, musi on posiadać pewną informację o tym, co i w jakiej formie należy w nim zapisać.  Program musi przeszukać szablon w celu znalezienia tej informacji i przeanalizowanie jej, aby wywołać odpowiednie procedury na rzecz danego szablonu. Jako, że informacja ta przeznaczona jest tylko dla programu przeszukującego, idealnym było by, gdyby zapis tej informacji byłby ignorowany przez środowisko kompilacji raportów.  W wybranym wcześniej środowisku \emph{Latex} znajduje się komenda \emph{iffalse} oraz jej zamknięcie \emph{fi} dzięki której wszystko pomiędzy zostanie zignorowane w czasie kompilacji dokumentów. Daje nam to taki zapis gdzie nasza informacja może zajmować wiele linii:
\begin{lstlisting}
\iffalse 
...
...
\fi
\end{lstlisting}
Jeśli polecenie jest w stanie zmieścić się w jednej linii, można użyć \% aby za komentować tę linię, co przyniesie dokładnie taki sam efekt.
\par
Kolejnym krokiem jest ustalenie struktury informacji, która powinna zawierać instrukcje do stworzenia wywołań środowisk z danymi przesłanymi jako parametry. Instrukcjami tymi są po kolei: nazwa środowiska, grupowanie pól oraz selekcja danych. Utworzony został więc na potrzeby tego programu standardowy zapis w różnych wariantach z wykorzystaniem znaków \emph{@} jako separatorów oraz informacji o zakończeniu instrukcji \emph{@END@}:\vspace{5mm}
\begin{enumerate}
\item Pusty - Polecenie do bazy danych, które nie zwraca żadnych danych.
\begin{lstlisting}
\iffalse 
@@@@Polecenie do bazy danych@END@
\fi
\end{lstlisting}

\item Prosty - Polecenie do bazy danych, zwracające wyselekcjonowane dane pod daną nazwą środowiska
\begin{lstlisting}
\iffalse 
@@Nazwa środowiska@@Selekcja danych@END@
\fi
\end{lstlisting}

\item Z grupowaniem - Polecenie do bazy danych, zwracające wyselekcjonowane dane pod daną nazwą środowiska dodatkowo z informacją o grupowaniu, która składa się z cyfr oraz przecinków.
\begin{lstlisting}
\iffalse 
@@Nazwa środowiska@Grupowanie@@Selekcja danych@END@
\fi
\end{lstlisting}
\end{enumerate}

\subsection{Selekcjonowanie danych}

Wybieranie danych z bazy odbywać się będzie na poziomie połączenia z wybranym silnikiem bazodanowym. Oznacza to ze zapytania o dane muszą zostać napisane tak, by interpreter poleceń SQL danego silnika był w stanie je przetworzyć i wykonać, zwracając przy tym potrzebne dane. Oznacza to, że zapytanie zapisane w danej instrukcji wywoływane jest bez żadnych zmian na bazie danych z którą połączona jest aplikacja.

\subsection{Struktura uzupełnianych danych}

Środowiska utworzone w szablonach są w stanie same, za pomocą argumentów, uzupełnić dane miejsca, o daną wartość. Zadaniem programu jest utworzyć z wyselekcjonowanych danych, wywołania tego środowiska jednorazowo dla każdego rekordu pobranego z bazy danych. Wywołanie środowiska odbywa się poprzez polecenie:
\begin{lstlisting}
\nazwasrodowiska{parametr1}{parametr2}{parametr3} ...
\end{lstlisting}

Taka struktura może być wynikiem prostego wariantu polecenia:
\begin{lstlisting}
\iffalse@@parametrRekrutacyjny@@
SELECT klucz,wartosc FROM setup_aligeza
@END@\fi
\end{lstlisting}
Gdzie wynikiem takiego polecenia będzie właśnie:
\begin{lstlisting}
\parametrRekrutacyjny{rokAkademicki}{2014/2015}
\parametrRekrutacyjny{czyUwzglednicDateWydaniaDecyzji}{N}
\parametrRekrutacyjny{instytutNazwa}{Instytut Techniczny}
\end{lstlisting}

Na potrzeby systemu rekrutacji, musiała zostać stworzona dodatkowa struktura, pełniąca funkcję grupowania. Wykorzystana może być także w przypadku gdy rekordy zwrócone z bazy danych zawierają więcej niż 9 pól, ze względu na to, że środowiska mogą być wywoływane maksymalnie od 9 argumentów. Poniżej przedstawiona zostanie tylko struktura z krótkim wprowadzeniem. Dokładny opis jej tworzenia znajduje się w implementacji.
\par
Strukturę grupowania odzwierciedla struktura drzewa. Wywołanie środowiska z dodaną dużą literą alfabetu łacińskiego na końcu nazwy rozpoczyna grupę, natomiast wywołanie środowiska z dodaną frazą "end" na początku nazwy, kończy daną grupę. Uwagę należy zwrócić na fakt, iż w alfabecie łacińskim jest 26 znaków, co ogranicza ilość grup do 26. Na przykładzie drzewa może wyglądać to następująco:

\dirtree{%
 .1 \textbackslash NazwaA.
 .2 \textbackslash NazwaB.
 .3 \textbackslash Nazwa.
 .3 \textbackslash Nazwa.
 .3 \ldots.
 .2 \textbackslash endNazwaB.
 .2 \textbackslash NazwaB.
 .3 \textbackslash Nazwa.
 .3 \textbackslash Nazwa.
 .2 \textbackslash endNazwaB.
 .1 \textbackslash endNazwaA.
}
\vspace{5mm}
Od każde grupujące środowisko wymaga co najmniej 1 parametru, który zabierany jest z pól rekordów pobranych z bazy danych. Pole dla wszystkich rekordów w danej grupie jest takie samo dlatego jest ono właśnie przerzucane do wywołania środowiska grupy.
Poniżej prosty przykład ukazujący dane zachowanie:
\begin{lstlisting}
\nazwasrodowiskaA{pole1}
\nazwasrodowiska{pole2}{pole3}{pole4}...
\nazwasrodowiska{pole2}{pole3}{pole4}...
...
\endnazwasrodowiskaA
\end{lstlisting}

\subsection{Wywołanie kompilacji szablonu}

Ostatecznie aby wytworzyć dokument w \emph{PDF} należy go skompilować wybranym kompilatorem. Do tego posłuży polecenie powłoki systemu Windows. Poniżej przykładowe wywołanie kompilatora LaTeX:
\begin{lstlisting}
cmd /c start texlive\2010min\bin\win32\pdflatex.exe 
--output-directory=output/ output/main.tex
\end{lstlisting}


\section{Wybór języka oraz środowiska programistycznego }

Pierwszym podstawowym kryterium wyboru języka programowania, w tym projekcie, jest fakt posiadania przez język gotowych bibliotek obsługujących połączenie z serwerem bazodanowym. Cała reszta wymagań takich jak obsługa operacji na plikach, operacje na łańcuchach tekstu, obiektowość języka czy też multiplatformowość schodzą na drugi plan, ze względu na to, że każdy współczesny język posiada większość podstawowych funkcjonalności. 
\par Na uczelni wykorzystywany jest serwer bazodanowy o silniku \emph{Firebird 2.5}. Najpopularniejsze języki programowania, które obsługują połączenie z tym serwerem to:
\begin{itemize}
\item JAVA
\item C++
\item C\#
\item Delphi
\item Perl
\item Python
\item wszystkie języki obsługujące połączenie z \emph{ODBC (Open DataBase Connectivity)} 
\end{itemize}
\vspace{5mm}
We wszystkich powyższych językach jest wstanie powstać potrzebny program, jednak najlepszym wyborem okazał się język \emph{\textbf{JAVA}}, ze względu na wiele zalet:
\begin{itemize}
\item prostota importu bibliotek połączenia z serwerem bazodanowym. Dodatkowo na jednym interfejsie można obsłużyć połączenia z innymi silnikami bazodanowymi.
\item multiplatformowość
\item obiektowość
\item wiele zaimplementowanych już funkcji, które zostaną wykorzystane w programie.
\item łatwość pisania kodu
\item wystarczająca wydajność na potrzeby projektu
\end{itemize}
\vspace{5mm}
\par
Natomiast na środowisko w jakim powstanie projekt wybrany został program \emph{\textbf{NetBeans 8.0.2 }} ze względu wiele przydanych funkcjonalności oraz  prostotę obsługi. Dodatkowo posiada on pełną dokumentację jak i wiele samouczków w Internecie. 

\section{Proces tworzenia programu}

W tej sekcji ukazany zostanie proces tworzenia aplikacji, a nie sama implementacja. Wyjaśnione będą decyzje oraz postępowania przy pisaniu danego kodu, aby w pełni oddać idee tworzenia tego programu. Sekcję tę, można więc potraktować jako pewien samouczek, który jednak wymaga minimalnej znajomości języka programowania \emph{JAVA}. Pominięte też zostaną kwestie związane z importem podstawowych bibliotek, czy też wszelkie obsługi wyjątków, które mogą zaistnieć przy złej obsłudze programu. Cała uwaga zostanie skupiona tylko i wyłącznie na kodzie, przy poprawnym wykonaniu. W razie potrzeby dokładnej analizy cały kod znajduje się w załączniku.

\subsection{Utworzenie projektu i jego struktury}

Środowisko \emph{NetBeans 8.0.2} posiada funkcję utworzenia projektu typu \emph{JAVA Application}. Automatycznie stworzony zostanie package o nazwie projektu \emph{DBLatexRaport} oraz klasa, o tej samej nazwie, zawierającą metodę \emph{main(String[] args)}. Od tej metody program zacznie swoje wykonanie. Tak więc, w metodzie tej, także będą umieszczane deklaracje wszystkich obiektów klas głównych oraz ich inicjalizacja. 
\par 
Aplikacja oprócz swojej głównej metody, będzie wymagała podziału na moduły, które będą odzwierciedlane poprzez odpowiednie klasy. Wymagane do działania będą:
\begin{itemize}
\item klasa Config - moduł obsługi pliku konfiguracyjnego 
\item klasa DBHandle - moduł obsługi połączenia z bazą danych.
\item klasa Templates - moduł obsługi szablonów.
\item klasa LatexCompiler -  moduł obsługi wywoływania kompilatora szablonów.
\end{itemize}
\vspace{5mm}
Podczas tworzenia programu może okazać się, że przydadzą się jeszcze dodatkowe klasy pomocnicze, przechowujące pewne dane w spójnej strukturze. Zastosowanie takich klas, znacznie ułatwi i przyspieszy pracę z danymi.Klasy te opisane zostaną w jednej z następnych podsekcji.

\subsection{Zarządzanie konfiguracją}

Większość programów przed uruchomieniem wymaga konfiguracji. Konfiguracje można przeprowadzić na wiele sposobów jednak najodpowiedniejszym sposobem do tej aplikacji będzie plik konfiguracyjny wczytywany przed uruchomieniem programu. Taki plik powinien posiadać informacje, z których skorzysta program, przypisane do zmiennych nazwany w stały sposób.  Dobrym zwyczajem jest aby plik konfiguracyjny posiadał też możliwość pisania komentarzy. Dzięki komentarzom, łatwiej jest skonfigurować dany program.
\par
Autorskim pomysłem jest dodatkowa funkcja pliku konfiguracyjnego, jaką jest aby plik ten był jednocześnie plikiem wsadowym. Polecenia powłoki systemu rozpoczynają plik, i kończą się na poleceniu \emph{EXIT}. Następnie od lini \emph{\#dbLatexRaportConfig} rozpoczyna się zapis zmiennych konfiguracyjnych.
\par
Klasa ta została tak zaprojektowana by mogła przyjąć wszelkie nazwy zmiennych oraz ich wartości w postaci łańcuchów znakowych, a następnie za pomocą odpowiedniej metody można było w środku programu odwołać się do wartości danej zmiennej po jej nazwie. Dla przykładu zdeklarowanie zmiennej za pomocą jej nazwy oraz zaraz po znaku "\emph{=}"  jej wartości:
\begin{lstlisting}
user=SYSDBA
\end{lstlisting}
Następnie po utworzeniu się obiektu klasy Config i załadowaniu poprawnie pliku konfiguracyjnego powinniśmy móc pobrać wartość  zmiennej \emph{user} poprzez wywołanie odpowiedniej metody. Oczywiście dla braku zmiennej, powinna zwracać pustą wartość i informować o tym użytkownika.
\begin{lstlisting}
Config cfg = new Config("dblatexraportconfig.bat");
System.out.println(cfg.getString("user"));
\end{lstlisting}

Podejściem do zaprojektowania tej klasy można uznać za metodę od ogółu do szczegółu.Na początek potrzeba zaimportować odpowiednie pakiety do obsługi plików, a następnie pozostaje już tylko ustalić składowe, zaimplementować sparametryzowany konstruktor klasy oraz metodę zwracającą daną zmienną. Składowe klasy muszą przede wszystkim przechowywać załadowane zmienne z pliku w pamięci. Łatwym sposobem realizacji tego jest użycie zmiennych tablicowych o typie \emph{String} czyli łańcuchów znaków. Dla każdego indeksu przechowywana będzie informacja o nazwie oraz wartości zmiennej. 
\begin{lstlisting}
String[] strindex;
String[] value; 
\end{lstlisting}

Następnie konstruktor klasy z jednym parametrem posłuży jako metoda otwarcia pliku i załadowania wszystkich zmiennych. Parametrem będzie nazwa pliku konfiguracyjnego. Konstruktor klasy w \emph{JAVA}'e musi nazywać się tak jak klasa. Po uruchomieniu konstruktora otwieramy plik wykorzystując dostępne klasy obsługi plików.
\begin{lstlisting}
public Config(String path){
 File cfg = new File(path);
 BufferedReader in = new BufferedReader(
 new InputStreamReader(
 new FileInputStream(cfg), "UTF-8"));
\end{lstlisting}
Do odczytu przygotować należy pewien bufor oraz zmienne pomocnicze:
\begin{lstlisting}
 String tmp;
 String[] strindextmp = new String[50];
 String[] valuetmp = new String[50];
 int count=0;
 int flag = 0;
\end{lstlisting}

Następnie potrzebna jest pętla odczytująca plik linia po linii. Pętla zakończy się po odczytaniu całego pliku, przekazując odczytaną linie do jednej iteracji w pętli. Wewnątrz pętli pierwszym krokiem jest sprawdzenie czy linia zawiera flagę \emph{\#dbLatexRaportConfig} by móc zacząć odczytywać zmienne poprzez ustawienie zmiennej flag na wartość 1. Pozostaje już tylko sprawdzić czy linia nie jest pusta oraz czy nie jest komentarzem, aby móc zapisać zmienne poprzez wyłączenie z linii nazwy (cała linia do znaku "=") oraz wartości zmiennej (cała linia od znaku "=") do aktualnego indeksu bufora. Ostatecznie potrzeba zwiększyć indeks bufora o 1 i zakończyć iteracje.
\begin{lstlisting}
while ((tmp = in.readLine()) != null) {
 if(flag == 0 && tmp.equals("#dbLatexRaportConfig"))
  flag = 1;
 if(flag == 1 && tmp.indexOf('#') != 0
 && tmp.length() != 0 && tmp.indexOf('=') != -1){
  strindextmp[count] = tmp.substring(0,tmp.indexOf('='));
  valuetmp[count] = tmp.substring(tmp.indexOf('=')+
  1,tmp.length());
  count++;
 }
}
\end{lstlisting}

Na koniec zamykany jest plik i alokowana jest pamięć o dokładnie \emph{count} elementów dla składowych klasy. Ostatecznie przepisywany jest cały bufor do tych składowych.
\begin{lstlisting}
in.close();
        
strindex = new String[count];
value = new String[count];
                  
for(int i=0;i < count; i++){
 strindex[i] = strindextmp[i];
 value[i] = valuetmp[i];
}
\end{lstlisting}

Pozostaje zaimplementowanie metody zawracającej wartość zapamiętanej zmiennej. Do tego posłuży metoda o nazwie \emph{getString} i parametrze \emph{String name}. Parametr jest kluczem do wartości zmiennej, która zostanie zwrócona.  Aby znaleźć tą wartość należy przejrzeć tablicę nazw zmiennych i porównać każdą wartość z nazwą szukanej zmiennej. Jeśli znajdziemy odpowiednik oznaczać to będzie, że pod tym samym indeksem w tablicy wartości znajduje się szukana zmienna, którą zwróci metoda. W przypadku braku zmiennej metoda zwróci pustą wartość i wyświetli informacje o jej braku.

\begin{lstlisting}
public  String getString(String name){
 for(int i=0;i < value.length; i++){
   if(strindex[i].indexOf(name) == 0)
    return(value[i]);
  }
 System.out.print("CONFIG VARIABLE ERROR: " + name + "\n");
 return("");
}
\end{lstlisting}

\subsection{Klasy typu kontener}

Przed przystąpieniem do tworzenia reszty głównych klas programu, potrzebne będzie stworzenie dwóch klas pomocniczych zawierających pewną strukturę danych oraz interfejs. Posłużą one do zapewnienia łatwej wymiany danych pomiędzy modułami. Pierwszą taką klasą będzie struktura reprezentująca informację o selekcji danych, wczytaną z szablonu. Klasa zostanie nazwana \emph{ParsedSQLInfo} i zawierać będzie następujące składowe wyczerpujące wszelkie informacje na temat selekcji:

\begin{lstlisting}
public class ParsedSQLInfo {
 String query;
 String name;
 String group;
 int index;
 String data;
\end{lstlisting}
Odpowiednio:
    \begin{itemize}
    \item \emph{query} - polecenie SQL
    \item \emph{name} - nazwa środowiska.
    \item \emph{group} - grupowanie
    \item \emph{index} - miejsce zapisu do pliku
    \item \emph{data} - wygenerowane dane dla tego zapytania.
    \end{itemize}

Do klasy został stworzony konstruktor sparametryzowany, który przyda się przy tworzeniu obiektu:
   
\begin{lstlisting}
public ParsedSQLInfo(String query, String name,
String group, int index) {
 this.query = query;
 this.name = name;
 this.group = group;
 this.index = index;
 this.data = "";
}
\end{lstlisting}

Oraz podstawowy interfejs do pobierania zmiany lub pobrania danych:
\begin{lstlisting}
public int getIndex() {return index;}
public String getQuery() {return query;}
public String getName() {return name;}
public String getGroup() {return group;}
public String getData() {return data;}
public void setIndex(int index) {this.index = index;}
public void setQuery(String query) {this.query = query;}
public void setName(String name) {this.name = name;}
public void setGroup(String group) {this.group = group;}
public void setData(String data) {this.data = data;}
\end{lstlisting}


Druga klasa typu kontener (\emph{RecordSet}) przechowywać będzie wyniki zapytań SQL w postaci listy rekordów. Każdy rekord natomiast będzie reprezentowany jako tablica zmiennych typu String o danej długości. Składowe wyczerpujące wszelkie informacje wyglądać będą następująco:

\begin{lstlisting}
ArrayList<String[]> val;
int m;
\end{lstlisting}

Jak w poprzedniej klasie, zaimplementowany został interfejs oraz konstruktor służący do zainicjowania obiektu ilością kolumn w rekordach:
 \begin{lstlisting}
RecordSet(int columncount){
val = new ArrayList<String[]>();
m = columncount;
}
 \end{lstlisting}

Interfejs wymagał by łatwo można było nadpisać daną wartość oraz pobrać w zależności od numeru rekordu oraz numeru pola. Zwrócić uwagę należy na to iż przy zapisie nowych rekordów, lista automatycznie się poszerzy. Dodatkowo dodana jest możliwość pobrania ilości rekordów czy też pól w 1 rekordzie.
 \begin{lstlisting}
void setVal(String str,int i, int j){
 while(val.size() <= i)
  val.add(new String[m]);
 val.get(i)[j]=str;
}
String getVal(int i, int j){return(val.get(i)[j]);}   

int get_rows(){return(val.size());}
int get_cols(){return(m);}
\end{lstlisting}

\subsection{Obsługa połączenia z bazą danych}

Stworzenie klasy obsługującej połączenie z bazą danych oraz wywołania poleceń na tej bazie. wymagało będzie zaimportowania dodatkowych bibliotek. W języku \emph{Java} znajduje się moduł \emph{ JDBC(TM) Database Access}, w którym dostępna jest klasa \emph{java.sql.DriverManager} pozwalająca załadować odpowiednią bibliotekę dla wybranego silnika bazodanowego, zwracając obiekt z takim samym interfejsem dla każdej możliwej bazy danych.   Obiekt z tym interfejsem pozwala podłączyć do się do źródła danych, takie jak bazy danych. Możliwe jest wysyłanie zapytań i poleceń aktualizacji do bazy danych. Ostatecznie obsługuje też pobieranie i przetwarzanie wyników otrzymanych z bazy danych w odpowiedzi na zapytania. 
\par
W przypadku pracy z serwerem \emph{Firebird 2.5} potrzebne będzie zaimportowanie do projektu odpowiedniego pliku \emph{jaybird-full-2.2.9.jar}, który został specjalnie przygotowany dla połączeń z tym serwerem. Jest on do pobrania z oficjalnej strony "\emph{http://www.firebirdsql.org/}". W razie potrzeby połączenie z innym rodzajem bazy danych, wystarczy w analogiczny sposób zaimportować biblioteki przygotowane przez twórców danego serwera a następnie tylko wywołać je poprzez klasę \emph{java.sql.DriverManager}.
\par
Posiadając zaimportowaną odpowiednią bibliotekę można rozpocząć implementację. Klasa obsługująca połączenie powinna posiadać metody do połączenia się z serwerem oraz metodę do wysłania zapytania i pobrania danych, które zwróci dane polecenie. Do przechowywania zwróconych danych użyta zostanie wcześniej stworzona klasa kontener \emph{RecordSet}. Próba nawiązania połączenia będzie odbywała się w czasie tworzenia obiektu w konstruktorze sparametryzowanym, zawierające parametry dotyczące połączenia, takie jak użytkownik, hasło, adres czy port.
\par
Składowe klasy powinny przechowywać obiekt obsługujący aktualne połączenie oraz obiekt obsługujący wysyłanie zapytań do serwera:
 \begin{lstlisting}
public class DBHandle {
 Connection con;
 Statement stmt; 
 \end{lstlisting}
  Zgodnie ze wcześniejszymi ustaleniami, nawiązanie połączenia odbywać się będzie w konstruktorze klasy. Do konstruktora przekazaną zostaną wszystkie informacje możliwe informacje na temat połączenia w postaci parametrów:
   \begin{lstlisting}
   public DBHandle(String engine,String hostname,
   String port, String dbpath,String encoding,
    String user,  String password){
   \end{lstlisting}
Odpowiednio:
    \begin{itemize}
    \item \emph{engine} - Nazwa silnika bazodanowego
    \item \emph{hostname} - adres serwera
    \item \emph{port} - port serwera
    \item \emph{dbpath} - ścieżka do bazy danych lub po prostu nazwa
    \item \emph{encoding} - Kodowanie po stronie serwera łańcuchów znakowych.
    \item \emph{user} - Nazwa użytkownika
    \item \emph{password} - hasło
    \end{itemize}

Następnie w ciele konstruktora nawiążemy połączenie poprzez stworzenie tak zwanego \emph{Connection String}, czyli łańcucha znaków w odpowiednim formacie zawierającego informacje na temat połączenia, które ma zinterpretować klasa \emph{java.sql.DriverManager} i spróbować nawiązać połączenie. \emph{Connection String} przyjmuje wiele zmiennych także w zależności od rodzaju połączenia. Poniżej pokazany zostanie przypadek nawiązania połączenia z serwerem \emph{Firebird 2.5} po adresie IP lub nazwie hosta. Najpierw ładowny jest sterownik \emph{org.firebirdsql.jdbc.FBDriver}, następnie tworzony jest \emph{Connection String} z parametrów przekazanych do konstruktora i ostatecznie wywoływana jest próba połączenia przez metodę \emph{getConnection()}:
   \begin{lstlisting}
if(engine.equals("firebirdsql")){
 try {
  Class.forName("org.firebirdsql.jdbc.FBDriver");
 } catch (ClassNotFoundException ex) {
  System.out.println( "Data Base connection driver error for 
  Firebird \nEXCEPTION: " + ex.getMessage( ) );
 }
  constr = "jdbc:" + engine + ":" + hostname + ":" + port + "/"
   + dbpath + "?encoding=" + encoding;
  con = DriverManager.getConnection(constr,user,password);
}
\end{lstlisting}

Na koniec jeśli połączenie się powiedzie tworzymy obiekt odpowiedzialny za wywoływanie zapytań na danym połączeniu i referencję do niego zapisujemy pod składową \emph{stmt}.
\begin{lstlisting}
stmt = con.createStatement(
ResultSet.TYPE_FORWARD_ONLY,
ResultSet.CONCUR_READ_ONLY);
System.out.println( "Connection to database: OK" );
}
\end{lstlisting}

Pozostaje zaimplementować metodę odpowiedzialną za wysyłanie zapytania i pobieranie danych na obecnym połączeniu. Metoda będzie przyjmować parametr w postaci zapytania SQL i zwracać obiekt klasy \emph{RecordSet} uzupełniony o wszystkie rekordy zwrócone przez zapytanie.
\begin{lstlisting}
public  RecordSet executeSQL2(String SQL){
\end{lstlisting}

Następnie deklaracje potrzebnych zmiennych oraz wywołanie polecenia na bazie danych. Wyniki zapytania zwracane są do klasy \emph{ResultSet}. Dodatkowo uzyskane zostaną jeszcze meta dane na temat ilości kolumn w pobranych rekordach. 
\begin{lstlisting}
int n=0,m=0,i=0,j=0;
String tmp;
RecordSet ret = null;
ResultSet rs = stmt.executeQuery( SQL );
rsmd = rs.getMetaData();
m=rsmd.getColumnCount();
\end{lstlisting}

Ze względu na brak odpowiedniego interfejsu ze strony obiektu klasy ResultSet na potrzeby dalszych manipulacji danymi, wszystkie rekordy zostaną przepisane do obiektu klasy \emph{RecordSet}. W razie wartości  \emph{null} jakiegoś pola w rekordzie, zastąpiona ona będzie łańcuchem znaków:
\begin{lstlisting}
while(rs.next()) {
 for(j=0;j < m; j++){
  tmp = rs.getString(j+1);
  if(tmp != null){
   ret.setVal(tmp, i, j);
  }
  else
   ret.setVal("null", i, j);
  }
 i++;
}   
rs.close();
return(ret);
}
\end{lstlisting}





\subsection{Zarządzanie szablonami}
test
\subsection{Klasa zarządzająca kompilatorem Latex}
test
\section{ Kompilacja programu}
test