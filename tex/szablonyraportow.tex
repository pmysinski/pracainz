\chapter{Szablony raportów w systemie Latex}
\label{ch:szablonyraportowwsystemielatex}
\section{Środowisko kompilacji raportów}

blablabla nie mam pojecia

\section{Idea działania szablonów}

Do wszystkich tych dokumentów potrzebny jest szablon w języku oprogramowania do zautomatyzowanego składu tekstu. W tej pracy został wybrany program LaTeX ze względu na jego możliwości automatyzacji procesu parsowania danych i uzupełniania nimi danych miejsc w tekście.
\par Stworzenie szablonów polega, więc na wcześniejszym przygotowaniu plików tex, zawierających wcześniej strukturę danego dokumentu z "pustymi" miejscami do uzupełnienia przez program. Do uzupełnienia tych miejsc można wykorzystać funkcję LaTeXu jaką jest tworzenie nowych środowisk z parametrami, gdzie odpowiednio parametry te będą wartościami, które zostaną wpisane w dane miejsce w danym dokumencie. Następnie wystarczy wywołać dane środowisko z odpowiednimi wartościami aby otrzymać uzupełniony dokument. Dane środowisko możemy wywoływać wielokrotnie od różnych wartości tworząc w ten sposób wiele dokumentów tego samego typu o różnych zmiennych wartościach takich jak np imię i nazwisko. 
\par Do wytworzenia wywołań tych środowisk posłuży właśnie program stworzony w javie. Poprzez dodanie zapytania SQL w odpowiedniej formule do plików tex. Program DBRaportLatex wyszuka takie zapytanie i uzupełni szablon wywołaniami środowisk z wartościami parametrów, jakimi będą wartości pola z rekordów zapytania SQL. 




\section{Tworzenie szablonów raportów do systemu rekrutacji}

W rekrutacji na uczelnie wykorzystuje się dokumenty, które należało dokładnie odwzorować w nowym systemie. Są to następujące dokumenty:
\begin{enumerate}
\item protokół przekazania
\item listy potwierdzenia podjęcia studiów 
\item listy rankingowe 
\item listy przyjętych
\item listy nieprzyjętych
\item decyzja o przyjęciu danego kandydata
\item decyzja o nieprzyjęciu danego kandydata
\end{enumerate}

Przy tworzeniu szablonów wystąpiły powtarzające się problemy, które należało rozwiązać.  Jako że rozwiązania tych problemów powtarzają się, to zamiast opisywania każdego szablonu po kolei, przedstawione poniżej zostały najważniejsze problemy, wynikające z tworzenia tych szablonów.

\subsection{Wyświetlanie listy}
Problem występujący w protokole przekazania oraz we wszystkich listach. Przykładowo potrzebujemy wyświetlić poniższą listę, gdzie oczywiście wartości pochodzą z bazy danych:
\begin{verbatim}
Lp. Tok studiów Liczba kopert
1 Informatyka — niestacjonarne STUDIA pierwszego stopnia 1455
2 Informatyka — stacjonarne STUDIA pierwszego stopnia 729
3 Mechatronika — niestacjonarne STUDIA pierwszego stopnia 1447
...
\end{verbatim}

