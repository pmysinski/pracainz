\chapter{Szablony raportów w systemie Latex}
\label{ch:szablonyraportowwsystemielatex}

W tym rozdziale przedstawiony zostanie proces przygotowania szablonów dokumentów potrzebnych przy rekrutacji na uczelni PWSZ Nowy Sącz. Opisane zostaną tylko problemy wynikające z tworzenia automatycznie uzupełnianych szablonów oraz ich rozwiązania.

\section{Idea działania szablonów}

Do wszystkich tych dokumentów potrzebny jest szablon w języku oprogramowania do zautomatyzowanego składu tekstu. W tej pracy został wybrany program LaTeX ze względu na jego możliwości automatyzacji procesu parsowania danych i uzupełniania nimi danych miejsc w tekście.
\par Stworzenie szablonów polega, więc na wcześniejszym przygotowaniu plików tex, zawierających wcześniej strukturę danego dokumentu z "pustymi" miejscami do uzupełnienia przez program. Do uzupełnienia tych miejsc można wykorzystać funkcję LaTeXu jaką jest tworzenie nowych środowisk z parametrami, gdzie odpowiednio parametry te będą wartościami, które zostaną wpisane w dane miejsce w danym dokumencie. Następnie wystarczy wywołać dane środowisko z odpowiednimi wartościami aby otrzymać uzupełniony dokument. Dane środowisko możemy wywoływać wielokrotnie od różnych wartości tworząc w ten sposób wiele dokumentów tego samego typu o różnych zmiennych wartościach takich jak np imię i nazwisko. 
\par Do wytworzenia wywołań tych środowisk posłuży właśnie program stworzony w javie. Poprzez dodanie zapytania SQL w odpowiedniej formule do plików tex. Program \emph{DBLatexRaport} wyszuka takie zapytanie i uzupełni szablon wywołaniami środowisk z wartościami parametrów, jakimi będą wartości pola z rekordów zapytania SQL. 

\section{Środowisko kompilacji raportów}

Środowisko do kompilacji dokumentów zostało specjalnie przygotowane poprzez usuniecie nadmiarowych (nieużywanych) bibliotek. Dzięki temu cały system będzie zajmował mniej pamięci na dysku twardym i będzie łatwiejsze do przenoszenia. Dodatkowo kompilatora nie trzeba instalować, dzięki czemu cały system będzie szybki w użyciu. Środowisko kompilacji wymaga systemu operacyjnego Windows. Środowisko to znajduje się w załączniku razem z programem.


\section{Tworzenie szablonów raportów do systemu rekrutacji}

W rekrutacji na uczelnie wykorzystuje się dokumenty, które należało dokładnie odwzorować w nowym systemie. Są to następujące dokumenty:\\
\begin{enumerate}
\item protokół przekazania
\item listy potwierdzenia podjęcia studiów 
\item listy rankingowe 
\item listy przyjętych
\item listy nieprzyjętych
\item decyzja o przyjęciu danego kandydata
\item decyzja o nieprzyjęciu danego kandydata 
\end{enumerate}
\vspace{5mm}
Przy tworzeniu szablonów wystąpiły problemy, które należało rozwiązać.  W dużej mierze problemy te powtarzają się,  opisane więc zostały tylko rozwiązania tych problemów a nie każdy szablon raportu. W poniższych podsekcjach znajdują się przedstawione problemy oraz ich rozwiązania.
\subsection{Definiowanie zmiennych}



\subsection{Wyświetlanie listy}
Problem występujący w protokole przekazania oraz we wszystkich listach. Przykładowo potrzebujemy wyświetlić poniższą listę, gdzie oczywiście wartości pochodzą z bazy danych:
\begin{lstlisting}
Lp. Tok studiów Liczba kopert
1 Informatyka — niestacjonarne STUDIA pierwszego stopnia 1455
2 Informatyka — stacjonarne STUDIA pierwszego stopnia 729
3 Mechatronika — niestacjonarne STUDIA pierwszego stopnia 1447
...
\end{lstlisting}

\subsection{Grupowanie}
\subsection{Aktualizacja daty}