\chapter{Instrukcja obsługi}
DBRaportLatex jest programem napisanym w JAVA'ie służącym do tworzenia raportów w środowisku latex z informacji przechowywanych w bazie danych. Program ten został przygotowany specjalnie w celu obsługi tworzenia dokumentów potrzebnych do rekrutacji na uczelni PWSZ Nowy Sącz.
Program oparty jest głównie o wcześniej przygotowane szablony w środowisku latex, które w czasie tworzenia raportów są uzupełniane informacjami pobranymi przez program z bazy danych.
Środowisko kompilacji szablonów zostało także przygotowane razem z programem. Kompilacja szablonów jest uruchamiana zaraz po uzupełnieniu szablonów i wynikiem jej jest plik pdf gotowy do wydruku.
\section{Wymagania systemowe}
Program do uruchomienia wymaga systemu operacyjnego z zainstalowanym pakietem \emph{Java SE Runtime Environment 8}.
\section{Przygotowanie pliku konfiguracyjnego}
Plik konfiguracyjny powinien mieć nazwę \texttt{dblatexraportconfig.bat}. Plik konfiguracyjny jest jednocześnie plikiem wsadowym który uruchamia program. Polecania wsadowe rozpoczynają plik i kończą się przy fladze \texttt{\#dbLatexRaportConfig}. Przykładowy początek pliku konfiguracyjnego:
\begin{lstlisting}
@echo off
java dblatexraport.jar
exit
#dbLatexRaportConfig
\end{lstlisting}
Po fladze \texttt{\#dbLatexRaportConfig} wprowadzane są zmienne. Wprowadza się  za pomocą nazwy zmiennej oraz jej wartości po znaku "=". Linie zaczynające się od \# zostaną zignorowane przez parser i mogą służyć jako komentarze. Na przykład:
\begin{lstlisting}
#Ścieżka do katalogu z szablonami
templatepath=template/

#Ścieżka do katalogu wynikowego: szablonami uzupełnionymi
# o dane z bazy danych oraz plików pdf
output=output/
\end{lstlisting}
Poniżej opisane są wszystkie zmienne:\vspace{5mm}
\begin{itemize}

\item \textbf{templatepath} - ścieżka do szablonów.
\item \textbf{output} - ścieżka do zapisu wyników pracy programu
\item  \textbf{encodingtex} - kodowanie zapisu szablonów tex
\item \textbf{pattern} - opcjonalnie, wyrażenie regularne przechwytujące polecania w szablonach.
\item  \textbf{dbengine} - nazwa sterowników do obsługi danego silnika bazy danych (obsługiwane firebirdsql, mysql, sqlite)
\item  \textbf{hostname} - IP lub nazwa hosta serwera bazy danych ( \texttt{//192.168.1.2 lub} \texttt{//Komputer-PC} ) lub dla lokalnego pliku firebird'a \texttt{embedded}.
\item  \textbf{port} - port serwera bazy danych
\item  \textbf{dbpath} - ścieżka do bazy danych w przypadku firebirda lub nazwa bazy danych w przypadku MySQL
\item  \textbf{user} - nazwa użytkownika
\item  \textbf{password }- hasło
\item  \textbf{dbencoding} - kodowanie znaków w bazie danych  (np. \texttt{dbencoding=WIN1250}  . Jeśli puste, nastąpi automatyczne wykrycie kodowania)
\item  \textbf{pdflatexpath} - ścieżka do kompilatora latex
\item  \textbf{pdfcompilemainfile} - zmienna określająca jakie pliki kompilować za pomocą kompilatora latex. Można tutaj użyć wartości \texttt{ALL} (wszystkie), \texttt{ONLYBEGDOC} (skompilowane zostaną dokumenty tylko z begindoc), \texttt{NONE} (żadne) lub wymienić nazwy plików po przecinku (\texttt{main.tex,setup.tex ...})

\end{itemize}


\section{Polecenia w szablonach}

Informacje zawarte w poleceniach do programu mają na celu wyselekcjonowanie danych z bazy danych i uzupełnienie szablonu zaraz po poleceniu o te pobrane dane. Proste polecenie zawiera w sobie:
\begin{itemize}
\item nazwę generowanego wywołania środowiska
\item zapytanie SQL
\end{itemize}
\par
Specjalnie dla tego programu został opracowany zapis polecenia tak aby kompilator dokumentów LaTeX ignorował to polecenie ale program mógł go przechwycić (w zapytaniu jest możliwość łamania linii):
\begin{lstlisting}
\iffalse 
@@Nazwa środowiska@@Zapytanie
SQL
do bazy danych@END@
\fi
\end{lstlisting}
Wynikiem takiego polecenia będzie następująca struktura dla każdego rekordu pobranego przez zapytanie SQL:
\begin{lstlisting}
\nazwasrodowiska{pole1}{pole2}{pole3}...
\end{lstlisting}
Ograniczenie parametrów przyjmowanych przez środowisko niestety powoduje iż w zapytaniu może być maksymalnie 9 zwracanych wartości.

\subsection{Polecenie puste}
Polecenie puste jest jedynie wywoływane na bazie danych i nie zwraca ono żadnych danych. Nawet jeżeli umieścimy w nim jakąś selekcje zwracającą dane to i tak nie zostaną one wpisane do szablonu. Polecenia tego można użyć do uaktualnienia np. daty w której przeprowadzamy generowanie dokumentów.
\par
Zapis polecenia pustego:
\begin{lstlisting}
\iffalse 
@@@@UPDATE .....@END@
\fi
\end{lstlisting}

\subsection{Grupowanie}
W programie istnieje także możliwość grupowania po polach, które zwraca zapytanie. Zapis grupowania jest niezależny od zapisu grupowania, natomiast działanie wymaga pewnego dodatku do zapytania jakim jest dodanie sortowania po tych polach(w SQL polecenie ORDER BY).\\
Zapis grupowania:
\begin{lstlisting}
\iffalse 
@@Nazwa środowiska@Grupowanie@@Selekcja danych@END@
\fi
\end{lstlisting}
W miejscu \texttt{Grupowanie} należy wpisać liczby oddzielone przecinkami, gdzie każda liczba będzie oznaczać grupę i ilość pobranych pól do tej grupy w zależności od wpisanej liczby. Poniżej przykład zapisu grupowania:
\begin{lstlisting}
\iffalse 
@@lista@1,2@@SELECT pole1, pole2, pole3, pole4@END@
\fi
\end{lstlisting}
\par 
Grupowanie działa według zasad:
\begin{itemize}
\item Z pobranych rekordów z bazy danych od lewej strony zabierane sa te pola, które sa grupowane i wytwarzane są z nich grupy ze wszystkich ich możliwych kombinacji ich wartości. 
\item Ostatecznie wywołania środowisk będą miały tylko te wartości pól które nie były grupowane.
\item Każda grupa będzie otwarta i zamknięta poprzez dodanie do nazwy środowiska end na początku.
\item Grupy nie przeplatają się.
\item Wszystkie rekordy objęte są wszystkimi grupami.
\item Nazwa grupy tworzona jest poprzez dodanie na końcu nazwy środowiska dużej litery łacińskiego alfabetu, w zależności która jest to grupa(Grupa pierwsza A, druga B ...). 
\item Ograniczenie ilości grup to 26.
\item Dla grupowanych pól wymagane jest dodanie sortowania (w SQL ORDER BY).
\end{itemize}

Struktura przykładowego grupowania:
\begin{lstlisting}
\listaA{pole1}{pole2}
\listaB{pole3}
\lista{pole4}
\lista{pole4}
...
\endlistaB
\listaB{pole3}
\lista{pole4}
\lista{pole4}
...
\endlistaB
\endlistaA
...
\end{lstlisting}


\section{Przygotowanie prostego szablonu}

Uwaga: Do tworzenia szablonów wymagana jest minimalna znajomość pracy w środowisku Latex, dlatego też przedstawiony zostanie proces tworzenia prostego szablonu bez wyjaśniania znaczenia i działania poszczególnych komend.
\vspace{5mm}
\par Program generuje z otrzymanych danych, wywołania komend w Latex'ie. Dlatego też, aby wykorzystać te dane należy stworzyć nową komendę lub środowisko, które przetworzy dane parametry.\\ Na przykład polecenie przyjmujące dwa argumenty, wyświetlające jeden wiersz w tablicy, dodatkowo posiadające licznik i wyświetlające licznik w pierwszej kolumnie a następnie oba argumenty w dwóch kolejnych kolumnach:
\begin{lstlisting}
\newcommand{\wiersz}[2]{\stepcounter{lp}\thelp&{\tiny #1}&#2\\}
\end{lstlisting}

Przykładowe dane wywołujące to polecenie będą wyglądać następująco:
\begin{lstlisting}
\lista{Przykładowy tekst}{Przykładowy  tekst w kolumnie 3}
\lista{Przykładowy tekst pojawi się w 2 wierszu}{Przykładowy tekst w kolumnie 3 wiersz 2}
\end{lstlisting}

Posiadając skonfigurowane polecenie należy utworzyć prosty szablon rozpoczynający tablicę:
\begin{lstlisting}
\begin{center}
\textbf{Przykładowa lista:} 
\end{center}

Poniżej powinna być tabela z wartościami numerowana od 1:
\begin{center}
\begin{tabular}{clc}
\hline
Lp.&\multicolumn{1}{c}{Nagłówek 2 kolumny}&Nagłówek 3 kolumny\\
\hline
\end{lstlisting}

Następnie w tym miejscu należy umieścić polecenie uzupełniające dane. Jak napisać takie polecenie opisane jest w poprzedniej sekcji. Teraz użyte zostanie proste polecenie pobierające 2 pola z bazy danych, zapisujące pod nazwą wywołania komendy lista:
\begin{lstlisting}
\iffalse @@lista@@
SELECT pole1, pole2
FROM tabela
@END@\fi
\end{lstlisting}
Polecenie to podczas kompilacji dokumentu zostanie zignorowane, wywołane zostaną tylko komendy które ów polecenie zwróciło podczas uzupełniania szablonów przez program.
\par
Ostatecznie należy zamknąć tabelę:
\begin{lstlisting}
\hline\end{tabular}\end{center}
\end{lstlisting}

W ten sposób otrzymany został prosty szablon tablicy uzupełnianej poprzez zapytanie do bazy danych i uzupełniane rekordami posiadającymi 2 pola z danej tabeli. Wynik uruchomienia programu i przeprowadzenia uzupełnienia szablonu wygląda następująco:
\begin{lstlisting}
\newcommand{\lista}[2]{\stepcounter{lp}\thelp&{\tiny #1}&#2\\}

\begin{center}
\textbf{Przykładowa lista:} 
\end{center}

Poniżej powinna być tabela z wartościami numerowana od 1:
\begin{center}
\begin{tabular}{clc}
\hline
Lp.&\multicolumn{1}{c}{Nagłówek 2 kolumny}&Nagłówek 3 kolumny\\
\hline
\iffalse @@lista@@
SELECT pole1, pole2
FROM tabela
@END@\fi
\lista{testtest}{123}
\lista{tekest teskt}{5555}
\lista{przykladowy tekst}{1111}

\hline\end{tabular}\end{center}
\end{lstlisting}

Natomiast wynik kompilacji powyższego szablonu przedstawiony został na rysunku poniżej.
\begin{figure}[h]
    \centering
    \label{fig:przykladszablon}
    \includegraphics[width=1\textwidth]{rys/instrukcja/szablon.png}
    \caption{Wynik kompilacji przykładowego szablonu przedstawiającego tabelę}
\end{figure}

\section{Uruchomienie programu}

Program można uruchomić za pomocą polecenia, które jednak najlepiej jest umieścić w pliku konfiguracyjnym (dblatexraport.bat):
\begin{lstlisting}
java -jar dblatexraport.jar
\end{lstlisting}

\subsection{Uruchomienie programu dla wersji embedded Firebird}

Wersja embedded dla silnika Firebird wymaga dodatkowego polecenia przy starcie programu:
\begin{lstlisting}
java -Djava.library.path=.\firebird -jar dblatexraport.jar
\end{lstlisting}
gdzie jako parametr \texttt{Djava.library.path} podajemy ścieżkę do rozpakowanej paczki zawierającej wszelkie potrzebne biblioteki pobranej ze strony \emph{http://www.firebirdsql.org/} dla wersji embedded.