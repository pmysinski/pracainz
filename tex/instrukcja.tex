\chapter{Instrukcja obsługi}
DBRaportLatex jest programem napisanym w JAVA'ie służącym do tworzenia raportów w środowisku latex z informacji przechowywanych w bazie danych.
Program ten został przygotowany specjalnie w celu obsługi tworzenia dokumentów potrzebnych do rekrutacji na uczelni PWSZ Nowy Sącz.
Program oparty jest głównie o wcześniej przygotowane szablony w środowisku latex, które w czasie tworzenia raportów są uzupełniane informacjami pobranymi przez program z bazy danych.
Środowisko kompilacji szablonów zostało także przygotowane razem z programem. Kompilacja szablonów jest uruchamiana zaraz po uzupełnieniu szablonów i wynikiem jej jest plik pdf gotowy do wydruku.
\section{Wymagania systemowe}
Program do uruchomienia wymaga zainstalowanego pakietu JAVA 1.8.
\section{Przygotowanie pliku konfiguracyjnego}
zmienne Wprowadza się  za pomocą nazwy zmiennej oraz jej wartości po znaku "=".
Poniżej są opisane wszystkie zmienne:

templatepath - ścieżka do szablonów.
templateoutput - ścieżka do uzupełnionych szablonów(tam zostaną zapisane)
encodingtex - kodowanie zapisu pliku *.tex

dbengine - nazwa sterowników do obsługi danego silnika bazy danych (obsługiwane firebirdsql, mysql)
hostname - IP lub nazwa hosta serwera bazy danych ( //192.168.1.2 lub //Komputer-PC )
port - port serwera bazy danych
dbpath - ścieżka do bazy danych w przypadku firebirda lub nazwa bazy danych w przypadku mysql
user - nazwa użytkownika
password - haslo
dbencoding - kodowanie znaków w bazie danych (może być puste)

pdflatexpath - ścieżka do kompilatora latex
pdflatexpathoutputfolder - ścieżka do skompilowanych plików (PDF)
pdfcompilemainfile - zmienna określająca jakie pliki kompilować za pomocą kompilatora latex. Można tutaj użyć wartości ALL (wszystkie), ONLYBEGDOC (skompilowane zostaną dokumenty tylko z begindoc), NONE (żadne) lub wymienić nazwy plików po przecinku (main.tex,setup.tex ...)
\section{Przygotowanie prostego szablonu}

\section{Polecenia w szablonach}
Uzupelnianie raportow polega na pobraniu przez program z bazy danych rekordow za pomoca przygotowanych wczesniej zapytan SQL zawartych w szablonyach.
Zapytania te nalezy umieszczac na poczatku szablonu w pliku *.tex. Kazde zapytanie to jedna linijka zaczynajaca sie od znaku procetów "%". 
Kazda kolejna linijka ze znakiem "%" bedzie traktowana jako zapytanie do bazy danych az do momentu nowej lini z innym znakiem.
Po tym znaku Piszemy nasze zapytanie z ktorego kazdy rekord bedzie wpisany do szablonu wg metody uzupelnienia ktora jest opisana ponizej.
Kazdy rekord wpisany pod koniec pliku lub miedzy znaczynikami startu i konca dokumentu jezeli one wystepuja w pliku bedzie wpisany wg szablonu:

'\' + nazwa pliku + duza litera numer 1(numer zapytania) + '{' + 1 pole rekordu 1 '}' + '{' + 2 pole rekordu 1  '}' + .. + '{' + n pole rekordu 1 '}'
'\' + nazwa pliku + duza litera numer 1(numer zapytania) + '{' + 1 pole rekordu 2 '}' + '{' + 2 pole rekordu 2  '}' + .. + '{' + n pole rekordu 2 '}'
...
'\' + nazwa pliku + duza litera numer 1(numer zapytania) + '{' + 1 pole rekordu m '}' + '{' + 2 pole rekordu m  '}' + .. + '{' + n pole rekordu m '}'

Dla kolejnych zapytan w jednym szablonie bedzie nadawana kolejna duza litera z alfabetu. Nalezy zauwazyc ze ilosc pól w zapytaniach moze sie roznic.

Przykladowo dla pliku o nazwie szablon(w tabeli vkandydat jest 6 pol i 2 rekordy, w setup 3 pola i 1 rekord):
%SELECT * FROM  vkandydat
%SELECT * FROM  setup

%\szablonA{Pawel}{Mysinski}{Przyjety}{1000}{STACJONARNE}{INFORMATYKA}
%\szablonA{Mateusz}{Mroz}{Przyjety}{1000}{STACJONARNE}{INFORMATYKA}

%\szablonB{2014}{jakis_text}{blablabla}


Ograniczenie zapytan w jednym szablonie jest jak widac do 26 tyle ile liter w alfabecie.
Ograniczenie pol dla zapytania jest narzucone przez Latex i jest to 9 pol natomiast program wyrzuci wiecej i moze to spowodowac problem przy kompilacji szablonu.
\subsection{Polecenie puste}

\subsection{Grupowanie}
Istnieje takze mozliwosc w programie grupowania po polach od ktorych zaczyna sie zapytanie. Zapis grupowania jest niezalezne od zapisu grupwania, natomiast dzialanie wymaga pewnego dodatku do zapytania jakim jest ordering.
Zapis grupowania:
%`a,b,c,d,...`SELECT ........ ORDER BY ....
'`' - nie mylic z apostrofem (tylda bez shift'a)
a,b,c,d,... - to liczby naturalne, ktore oznaczaja na ilu polach bedzie oparta grupa. grup moze byc od 1 do 26 i oparta maksymalnie na 9 polach.
ORDER BY .... - Zaznaczyc trzeba ze przed grupowaniem nalezy uporzatkowac rekordy wedlug pol po ktorych grupujemy inaczej grupy moga sie powtorzyc co zpowoduje zly wydruk danych.
Grupowanie dziala wedlug zasad:
Z pobranych rekordow z bazy danych zabierane sa te pola ktore sa grupowane i wrzucane do danej grupy o unikalnych swoich wartosciach. 
Ostateczne rekordy beda mialy tylko te wartosci pol ktore nie byly grupowane.
Kazda grupa bedzie otwarta i zamknieta.
Grupy nie przeplataja sie.
Zapytania z grupowaniem sa niezalezne tak samo jak kazde inne zapytanie i mozna je mieszac z innymi zapytaniami.

Struktura odnosnie nazewnictwa odnosnie ktore z kolei jest to zapytanie jest zachowana. Dodana jest natomiast struktura grup:
%nazwy dla poczatku grupy: '\' + nazwapliku + duza litera (wg numer zapytania)+ duza litera wg nr grupy +  "{" + wartosc 1 po zgrupwaniu + "}" + ... +"{" + wartosc n po zgrupwaniu + "}"
%Snazwy dla konca grupy: '\' + "end" + nazwapliku + duza litera (wg numer zapytania) + duza litera wg nr grupy +  "{" + wartosc 1 po_zgrupwaniu + "}" + ... +"{" + wartosc n po zgrupwaniu + "}" 
\section{Uruchomienie programu}
\section{Rozwiązywanie problemów}
