\chapter{Uruchomienie oraz testowanie Systemu}

Przed wdrożeniem programu do realnego systemu, program należy przetestować. Testy powinny być prowadzone na tymczasowej bazie danych, ponieważ idea testów jest taka, że po podmianie bazy danych na realną wszystko ma działać bez zmian. Zmienić się może tylko zapis połączenia z bazą danych w pliku konfiguracyjnym. Dzięki takiemu zabiegowi, będzie można być pewnym tego, że wszystko będzie działać na prawdziwej bazie danych.  Do przeprowadzenia testów potrzebne będzie  odtworzyć przyszłe środowisko, w którym będzie działać program, przygotować szablony dokumentów, które są wytwarzane w procesie rekrutacji oraz wygenerować dokumenty. Ostatnim już krokiem będzie sprawdzenie czy podczas tego procesu nie ma żadnych komplikacji oraz czy wygenerowane dokumenty nie zawierają błędów.

\section{ Generowanie przykładowych danych}

W celu przetestowania systemu generowania raportów w procesie rekrutacji kandydatów na studia potrzebne będą testowe dane w dokładnie tej samej strukturze co w systemie rekrutacji, ponieważ w szablonach latexu znajdują się zapytania SQL do danych tabel w bazie danych. 
W celu otrzymania tych danych potrzebne będzie:
1.	Utworzenie nowej bazy danych na silniku Firebird’a 
2.	Stworzenie wystarczającej struktury tabel odzwierciedlającą strukturę w systemie uczelnianym.
3.	Wygenerowanie dużej ilości testowych danych osobowych.
4.	Uzupełnienie tabel danymi, które zostały wygenerowane wcześniej oraz dodanie do nich dodatkowych, jednocześnie losowych, informacji na temat procesu rekrutacji.
Po wykonaniu tych kroków, powinna powstać baza danych do której bez problem program połączy się i wyciągnie z niej potrzebne dane dokładnie jak z realnej bazy danych.

\subsection{ Tworzenie bazy danych }

Do stworzenia pliku bazy danych na silniku firebird’a posłużyć się można narzędziem dostepnym w katalogu bin zainstalowanego serwera firebird.  Narzędzie to pozwala z linii komend tworzyć i łączyć się z bazami danych. W tym przypadku użyta zostanie komenda „CREATE DATABASE”
\begin{lstlisting}
C:\Program Files\Firebird\bin>isql
SQL>CREATE DATABASE 'D:\test systemu\rekrutacja.fdb'
CON>user 'SYSDBA' password 'masterkey';
\end{lstlisting}
Po wykonaniu tego polecenia zostanie utworzona baza danych. Takie same dane należy teraz wpisać do pliku konfiguracyjnego programu czyli DBRaportLatex.bat aby program mogł się połączyć z tą bazą:
\begin{lstlisting}
dbengine=firebirdsql
hostname=//localhost
port=3050
dbpath=D:\test systemu\rekrutacja.fdb
user=SYSDBA
password= masterkey
\end{lstlisting}
\subsection{ Stworzenie struktury bazy}

Dotychczasowy system wykorzystywał tabelę (widok) która była generowana dynamicznie i która zawiera wszystkich studentów w rekrutacji. Zawiera ona wszystkie dane potrzebne do wytworzenia dokumentów. Jeden rekord to jeden student ze wszystkimi informacjami na jego temat. Dodatkowo jeszcze potrzebna jest tabela z informacjami na temat rekrutacji, takimi jak na przykład nazwisko i imię przewodniczącego komisji, czy data wydania decyzji przyjęcia studenta. Z tych tabel będą pobierane informacje, natomiast do wygenerowania danych potrzebne będą dwie dodatkowe tymczasowe tabele.
Tabela główna z kandydatami(zapis SQL):
\begin{lstlisting}
CREATE TABLE KANDYDAT_ALIGEZA (
    STUD_ID                        INTEGER NOT NULL,
    OSOBA_ID                       INTEGER NOT NULL,
    STUD_NRTECZKI                  INTEGER,
    NAZWISKO                       VARCHAR(100),
    IMIE                           VARCHAR(100),
    NAZWISKOIMIONA                 VARCHAR(200),
    ADR_ULICA_MIEJSCOWOSC_NR_DOMU  VARCHAR(200),
    ADR_KOD_POCZTOWY_POCZTA        VARCHAR(100),
    STUDIA_NAZWA                   VARCHAR(100),
    STUD_ILPUNKTOW                 INTEGER,
    STUD_ILPUNKTOWKREM             INTEGER,
    TOKNAUKI_NAZWATOKU             VARCHAR(200),
    KIERUNEK                       VARCHAR(100),
    SPEC_ID                        INTEGER,
    DATAPRZYJECIAPODANIA           DATE,
    TOKNAUKI_ID                    INTEGER,
    OSOBA_PESEL                    VARCHAR(50),
    KIERUNEK_MY                    VARCHAR(200),
    FORMA_STUDIOW_MY               VARCHAR(200),
    STOPIEN_STUDIOW_MY             VARCHAR(100),
    KIERUNEK_FORMA_SKROT_MY        VARCHAR(100),
    NR_DECYZJI                     VARCHAR(100),
    CZY_PRZYJETY                   INTEGER,
    DATA_DECYZJI                   DATE,
    ILE_PUNKTOW                    INTEGER,
    PANPANI                        CHAR(1)
);
\end{lstlisting}
Tabela z dodatkowymi informacjami(wartości przypisane są do kluczy tekstowych, jest to tablica asocjacyjna):
\begin{lstlisting}
CREATE TABLE SETUP_ALIGEZA (
    KLUCZ    VARCHAR(50) NOT NULL,
    WARTOSC  VARCHAR(100)
);
\end{lstlisting}
Tymczasowa tabela do zaimportowania listy imion i nazwisk oraz losowych peseli.
\begin{lstlisting}
CREATE TABLE DANE (
    IMIE_NAZ  VARCHAR(200),
    ADRES     VARCHAR(200),
    PESEL     VARCHAR(50),
    NAZWISKO  VARCHAR(100),
    IMIE      VARCHAR(100)
);
\end{lstlisting}
Tymczasowa tabela do procedury losowego uzupełniania informacji o rekrucie o kierunku jaki wybrał.
\begin{lstlisting}
CREATE TABLE TOKNAUKI_ALIGEZA (
    TOKNAUKI_ID              INTEGER NOT NULL,
    KIERUNEK_MY              VARCHAR(50),
    FORMA_STUDIOW_MY         VARCHAR(50),
    STOPIEN_STUDIOW_MY       VARCHAR(50),
    KIERUNEK_FORMA_SKROT_MY  VARCHAR(10),
    LICZBA_MIEJSC            SMALLINT,
    DATA_DECYZJI_OD          TIMESTAMP,
    DATA_DECYZJI_DO          TIMESTAMP,
    KOD_IKR                  VARCHAR(3)
);
\end{lstlisting}
\subsection{ Generowanie testowych danych osobowych}

Do wygenerowania kandydatów potrzeba imienia nazwiska oraz adresu. Takie dane dostępne są w książkach telefonicznych. Posługując się jedną z takich książek stworzony został plik csv o separatorze „;” zawierający po kolei imię z nazwiskiem, adres, pesel, nazwisko oraz imie. Pesel został dodany do każdej osoby jako losowy ciąg cyfr spełniający walidację peselu. Ze względu na fakt iż pesel został wygenerowany losowo, może zdarzyć się iż mężczyzna posiadać będzie kobiecy pesel, w następstwie czego, we wygenerowanych dokumentach wypisane zostanie „Pani” i na odwrót.
Struktura pliku:
\begin{lstlisting}
 Abram Andrzej; Lwowska 116;88071640299;Abram;Andrzej
 Abram Halina; Ludwika Zamenhofa 2;86111210691;Abram;Halina
...

\end{lstlisting}

Tak sformatowany plik CSV,  łatwo zaimportować do bazy danych do tabeli „dane” ze względu na identyczną kolejność danych w kolumnach. Do importu wykorzystana została funkcja programu IBExpert „import data”. Jedna linijka w pliku zostaje zaimportowana jako jeden rekord, w którym każde pole po kolei odpowiada wartościom między średnikami. Zaimportowanych w ten sposób zostało 10001 rekordów (osób) do tabeli „dane” do dalszych manipulacji.

\subsection{Generowanie kandydatów na studentów}

Kolejnym krokiem jest uzupełnienie tabeli z tokami studiów. W testach dodanych zostało 8 przykładowych toków nauki. 

\begin{lstlisting}
1 Informatyka	niestacjonarne	pierwszego stopnia	INF-n				
2 Mechatronika	niestacjonarne	pierwszego stopnia	MT-n				
3 Mechatronika	stacjonarne	pierwszego stopnia	MT-s				
...


\end{lstlisting}
Uzupełnienia wymaga także tabela z dodatkowymi informacjami „SETUP\_ALIGEZA” przykładowymi danymi:
\begin{lstlisting}
dataWydaniaDecyzji	09.10.2015
miejsceWydaniaDecyzji	Nowy Sącz
przewodniczacyIKR	mgr inż. Sławomir Jurkowski
rokAkademicki	2015/2016
czyUwzglednicDateWydaniaDecyzji	N
...

\end{lstlisting}

Mając już to wszystko potrzebna jest procedura, która utworzy listę kandydatów z tych wszystkich danych. 

\begin{lstlisting}
create procedure GENERUJ
returns (
    TESTCHAR varchar(50),
    TEST integer)
as
declare variable IMIE varchar(100);
declare variable NAZ varchar(100);
declare variable IMIENAZ varchar(200);
declare variable ADRES varchar(200);
declare variable PESEL varchar(50);
declare variable LICZNIK integer;
declare variable STOPIEN varchar(50);
declare variable KIERUNEK varchar(50);
declare variable FORMA varchar(50);
declare variable SKROT varchar(10);
declare variable RANDINT integer;
declare variable PUNKTY integer;
declare variable CZY_PRZYJETY integer;
declare variable DATA_DEC varchar(100);
begin
licznik = 1;
for select * from dane into
:imienaz,:adres,:pesel,:naz,:imie
do
begin
randint = CAST(round(rand()*7+1) as INTEGER);
punkty = CAST(round(rand()*500) as INTEGER);
if(punkty > 250) then czy_przyjety = 1;
if(punkty <= 250) then czy_przyjety = 2;

select kierunek_my,forma_studiow_my,stopien_studiow_my,
kierunek_forma_skrot_my
FROM toknauki_aligeza where toknauki_id = :randint
INTO :kierunek,:forma,:stopien,:skrot;

select wartosc FROM setup_aligeza 
WHERE klucz='dataWydaniaDecyzji'
INTO :data_dec;

INSERT INTO kandydat_aligeza
(stud_id,osoba_id,stud_nrteczki,nazwisko,imie,
nazwiskoimiona,adr_ulica_miejscowosc_nr_domu,
adr_kod_pocztowy_poczta,osoba_pesel,panpani,
studia_nazwa,toknauki_nazwatoku,kierunek,
kierunek_my,forma_studiow_my,stopien_studiow_my,
kierunek_forma_skrot_my,    stud_ilpunktow,
stud_ilpunktowkrem,ile_punktow,czy_przyjety,
data_decyzji,nr_decyzji,dataprzyjeciapodania)
VALUES (:licznik,:licznik,cast(round(rand()*200+1) as integer),
:naz,:imie,:imienaz,:adres, 
cast( 'Nowy Sacz 33-300' as varchar(100)),
:pesel,cast( 'M' as char(1)),    :forma,:kierunek || 
' N inz. 3.50 2015/2016 zimowy',:kierunek,:kierunek,:forma,
:stopien,:skrot,    :punkty,:punkty,:punkty,:czy_przyjety,
cast(:data_dec as DATE),'328/2015','2015-08-14');

licznik = :licznik + 1;
end
test = :licznik;
suspend;
end
\end{lstlisting}
Powyższa procedura z jednej osoby z tabeli dane tworzy jednego kandydata, losując mu tok nauki, ilość punktów oraz czy zostanie przyjęty lub nie. Dorzucane są także pewne stałe wartości, podobne do tych w oryginalnej bazie danych, które nie wymagają uzmiennienia. Procedura ta, po jednorazowym wywołaniu, wygenerowała 10001 rekordów w tabeli "kandydat\_aligeza". Daje to wystarczającą ilość testowych kandydatów do przeprowadzenia testów. Baza danych z przeprowadzonych testów znajduje się w załącznikach.

\section{Dostosowanie zapytań SQL w szablonach}

Dokładny opis zapytań SQL i procesu ich tworzenia.


\section{Przebieg generowania raportów}

Proces uruchamiania oraz przebiegu

\section{Wyniki testu}

Krótki opis wyników testu.


