\chapter{Podsumowanie}
Opracowany system generowania raportów usprawni i przyspieszy wszelkie procesy, w których wymagana jest duża ilość dokumentów, miedzy innymi właśnie proces rekrutacji na uczeni PWSZ w Nowym Sączu.
\vspace{5mm}
\par
Dzięki programowi DBLatexRaport będzie możliwe:\vspace{5mm}
\begin{itemize}
\item pełna automatyzacja procesu tworzenia dokumentów oraz raportów przy pomocy wcześniej stworzonych szablonów.\vspace{5mm}
\item usprawnienie pracy pracownikom przy segregowaniu raportów\vspace{5mm}
\item znaczny wzrost prędkości przebiegu całego procesu.\vspace{5mm}
\item eliminacja błędów człowieka, które mogły powstać przy ręcznym tworzeniu raportów.\vspace{5mm}
\item ze względu na kompatybilność z wieloma silnikami bazodanowymi, uruchomienie programu z różnych źródeł danych.\vspace{5mm}
\item ze względu na fakt iż program jest napisany w języku JAVA 1.8, uruchomienie go na wielu systemach operacyjnych, które mają możliwość zainstalowania pakietu JAVA.\vspace{5mm}
\end{itemize}\vspace{5mm}
\par
Do programu dołączona jest krótka instrukcja obsługi, dzięki której każdy będzie w stanie skonfigurować i uruchomić program. Jedynym wymaganiem jest umiejętność pisania dokumentów w systemie LaTeX aby móc stworzyć szablony raportów.
\par 
Biorąc pod uwagę powyższe stwierdzenia, można uznać, że osiągnięto postawiony cel w pracy.