\chapter{Podsumowanie}
Opracowany w pracy system generowania raportów usprawni i przyspieszy wszelkie procesy, w których możliwa jest automatyzacja tworzenia dokumentów. Rozpoczęcie tworzenia systemu było głównie z myślą o obsłudze rekrutacji na uczelni PWSZ Nowy Sącz na wydziale Technicznym, jednak w czasie realizacji projektu koncepcja ta została zastąpiona pełnym uogólnieniem możliwości do których może on służyć. Dzięki temu stworzony system można wykorzystać, w każdym przypadku generowania dokumentów, w którym możliwe jest utworzenie szablonu w środowisku LaTeX.
\vspace{5mm}
\par
Projekt wcześniej przeszedł pozytywnie testy na wygenerowanych danych, które w wyczerpujący sposób sprawdziły funkcjonalność programu, co dało pełną gwarancję niezawodności przy zastosowaniu go przy tworzeniu dokumentów podczas rekrutacji na uczelni.
\vspace{5mm} 

Dzięki opracowanym szablonom dokumentów rekrutacyjnych, opracowany system został wykorzystany i sprawdzony w procesie rekrutacji kandydatów w PWSZ Nowy Sącz na wydziale Technicznym, w dwóch rekrutacjach zimowej i letniej 2015 roku. Potwierdził swoją przydatność i szybkość przeprowadzenia i opracowania wyników rekrutacyjnych. 
\vspace{5mm}
\par
Dzięki programowi DBLatexRaport stworzonemu na potrzeby pracy jest możliwe:\vspace{5mm}
\begin{itemize}
\item pełna automatyzacja procesu tworzenia dokumentów oraz raportów przy pomocy wcześniej stworzonych szablonów;\vspace{5mm}
\item usprawnienie pracy przy sporządzaniu raportów;\vspace{5mm}
\item znaczny wzrost prędkości przebiegu całego procesu;\vspace{5mm}
\item eliminacja błędów człowieka, które mogły powstać przy ręcznym tworzeniu raportów;\vspace{5mm}
\item ze względu na kompatybilność z wieloma silnikami bazodanowymi, uruchomienie programu z różnych źródeł danych (jeśli źródło danych posiada bibliotekę JDBC);\vspace{5mm}
\item ze względu na fakt iż program jest napisany w języku JAVA 1.8, uruchomienie go na wielu systemach operacyjnych, które mają możliwość zainstalowania pakietu JAVA;\vspace{5mm}
\end{itemize}\vspace{5mm}

Do programu dołączona jest krótka instrukcja obsługi, dzięki której każdy będzie w stanie skonfigurować i uruchomić program. Jedynym wymaganiem sprawnego użytkowania programu jest umiejętność pisania dokumentów w systemie LaTeX, wymagana do przygotowania szablonów raportów.
\vspace{5mm}

Biorąc pod uwagę powyższe stwierdzenia, można uznać, że osiągnięto postawiony w pracy, uzyskując ponadto uniwersalny system do generowania raportów w różnych dziedzinach.