\pdfoutput=1
\pdfcompresslevel=9
\pdfinfo
{
    /Author (Autor)
    /Title (Tytul)
    /Subject (Tematyka)
    /Keywords (Slowa kluczowe)
}
%\documentclass[a4paper,polish,onecolumn,oneside,floatssmall,11pt,titleauthor,wide,openright]{mwrep}
%\usepackage[scale={0.7,0.8},paper=a4paper,twoside]{geometry}
\documentclass[a4paper,onecolumn,oneside,12pt,wide,floatssmall]{mwrep}
% \usepackage{polish}
\usepackage{amsmath}
\usepackage{amsfonts}
\usepackage{amssymb}
\usepackage{amsthm}
\usepackage{bookman}
\usepackage{geometry}
\usepackage[utf8x]{inputenc}
\usepackage[T1]{fontenc}
% \usepackage{t1enc}
% \usepackage[pdftex, bookmarks]{hyperref}
\usepackage[pdftex, bookmarks=false]{hyperref}
\def\url#1{{ \tt #1}}

\usepackage{listings}

% marginesy
\textwidth\paperwidth
\advance\textwidth -55mm
\oddsidemargin-0.9in
\advance\oddsidemargin 30mm
\evensidemargin-0.9in
\advance\evensidemargin 30mm
\topmargin -1in
\advance\topmargin 20mm
\setlength\textheight{48\baselineskip}
\addtolength\textheight{\topskip}
\marginparwidth15mm
\newgeometry{tmargin=2.5cm, bmargin=2.5cm, lmargin=3cm, rmargin=2.5cm}

\clubpenalty=10000 % to kara za sierotki
\widowpenalty=10000 % nie pozostawia wdów
\brokenpenalty=10000 % nie dzieli wyrazów pomiędzy stronami
\sloppy

\tolerance4500
\pretolerance250
\hfuzz=1.5pt
\hbadness1450

\usepackage[pdftex]{color,graphicx}
\usepackage[polish]{babel}
\usepackage[T1]{fontenc}
\usepackage{mathptmx}
\usepackage{dirtree}

% \textheight232mm
% \setlength{\textwidth}{\textwidth}
% \setlength{\oddsidemargin}{\evensidemargin}
% \setlength{\evensidemargin}{0.3cm}
\usepackage[sort, compress]{cite}

%\usepackage{multibib}
%\newcites{bk,st,doc,web}{Książki i~artykuły,Standardy i~zalecenia,Dokumentacja produktów,Publikacje i~serwisy internetowe}

\theoremstyle{definition}
\newtheorem{defn}{Definicja}[section]
\newtheorem{conj}{Teza}[section]
\newtheorem{conjmain}{Teza}
\newtheorem{exmp}{Przykład}[section]

\theoremstyle{plain}% default
\newtheorem{thm}{Twierdzenie}[section]
\newtheorem{lem}[thm]{Lemat}
\newtheorem{prop}[thm]{Hipoteza}
\newtheorem*{cor}{Wniosek}

\theoremstyle{remark}
\newtheorem*{rem}{Uwaga}
\newtheorem*{note}{Uwaga}
\newtheorem{case}{Przypadek}

\definecolor{ListingBackground}{rgb}{0.95,0.95,0.95}

\lstset{
        language=SQL,
    inputencoding=utf8x, 
    extendedchars=\true,
    literate={ą}{{\k{a}}}1
             {Ą}{{\k{A}}}1
             {ę}{{\k{e}}}1
             {Ę}{{\k{E}}}1
             {ó}{{\'o}}1
             {Ó}{{\'O}}1
             {ś}{{\'s}}1
             {Ś}{{\'S}}1
             {ł}{{\l{}}}1
             {Ł}{{\L{}}}1
             {ż}{{\.z}}1
             {Ż}{{\.Z}}1
             {ź}{{\'z}}1
             {Ź}{{\'Z}}1
             {ć}{{\'c}}1
             {Ć}{{\'C}}1
             {ń}{{\'n}}1
             {Ń}{{\'N}}1
}

\begin{document}

\renewcommand*\lstlistingname{Wydruk}
\renewcommand*\lstlistlistingname{Spis wydruków}

\pagenumbering{roman}
\renewcommand{\baselinestretch}{1.0}
\raggedbottom
\input{tex/tytul.tex}

\newpage
\pagenumbering{arabic}
\setcounter{page}{3}

\tableofcontents
% \addcontentsline{toc}{chapter}{{Przedmowa1}{vii}}{vii}

% \chapter*{Spis tablic, rysunków i~wydruków}
% \listoftables
% \listoffigures
% \lstlistoflistings

%\setlength{\baselineskip}{7mm}


\chapter{Wprowadzenie}

Dokumenty, listy, protokoły to nierozłączna część każdego procesu rekrutacji na uczelni wyższej. Ręczne tworzenie takich dokumentów może nastręczać wielu problemów. Przede wszystkim głównym problemem jest możliwość wystąpienia błędów człowieka. W nawet najlepiej zorganizowanej placówce szkolnej może wkraść się błąd, który spowoduje iż na przykład kandydat zostanie przypadkowo odrzucony lub przyjęty na studia. Odkręcenie takich błędów może być czasami nie możliwe, dlatego też do przy ręczne tworzenie takich dokumentów zajmuje wiele czasu aby mieć pewność aby nie popełnić błędu. Przy czym istotnym jest, że rekrutacja powinna przebiegać szybko ze względu na fakt, iż kandydaci potrzebują znać decyzję o przyjęciu w miarę szybko aby móc w razie czego zgłosić się do innych szkół w tym samym okresie. Dodać jeszcze należy, że podczas nieefektywnie zorganizowanej rekrutacji, w procesie tym udział muszą wziąć pracownicy dydaktyczni, którzy muszą poświęcać swój czas na okres rekrutacji.


\section{ Zagadnienie generowania raportów}

Z rozwiązaniem tego problemu przychodzi informatyzacja procesu rekrutacji. Pozwala ona na eliminacje praktycznie wszystkich błędów człowieka poprzez automatyczne korekty danych oraz auto uzupełnianie dokumentów danymi. Przyspiesza proces rekrutacji poprzez ułatwienie go dla osób prowadzących go oraz fakt iż dokumenty generowane są w bardzo krótkim czasie.


\section{ Dotychczasowy proces generowania raportów}

\section{Cel i zakres pracy}


\chapter{Szablony raportów w systemie Latex}
\label{ch:szablonyraportowwsystemielatex}

W tym rozdziale przedstawiony zostanie proces przygotowania szablonów dokumentów potrzebnych przy rekrutacji na uczelni PWSZ Nowy Sącz. Opisane zostaną tylko problemy wynikające z tworzenia automatycznie uzupełnianych szablonów oraz ich rozwiązania.

\section{Idea działania szablonów}

Do wszystkich tych dokumentów potrzebny jest szablon w języku oprogramowania do zautomatyzowanego składu tekstu. W tej pracy został wybrany program LaTeX ze względu na jego możliwości automatyzacji procesu parsowania danych i uzupełniania nimi danych miejsc w tekście.
\par Stworzenie szablonów polega, więc na wcześniejszym przygotowaniu plików tex, zawierających wcześniej strukturę danego dokumentu z "pustymi" miejscami do uzupełnienia przez program. Do uzupełnienia tych miejsc można wykorzystać funkcję LaTeXu jaką jest tworzenie nowych środowisk z parametrami, gdzie odpowiednio parametry te będą wartościami, które zostaną wpisane w dane miejsce w danym dokumencie. Następnie wystarczy wywołać dane środowisko z odpowiednimi wartościami aby otrzymać uzupełniony dokument. Dane środowisko możemy wywoływać wielokrotnie od różnych wartości tworząc w ten sposób wiele dokumentów tego samego typu o różnych zmiennych wartościach takich jak np imię i nazwisko. 
\par Do wytworzenia wywołań tych środowisk posłuży właśnie program stworzony w javie. Poprzez dodanie zapytania SQL w odpowiedniej formule do plików tex. Program \emph{DBLatexRaport} wyszuka takie zapytanie i uzupełni szablon wywołaniami środowisk z wartościami parametrów, jakimi będą wartości pola z rekordów zapytania SQL. 

\section{Środowisko kompilacji raportów}

Środowisko do kompilacji dokumentów zostało specjalnie przygotowane poprzez usuniecie nadmiarowych (nieużywanych) bibliotek. Dzięki temu cały system będzie zajmował mniej pamięci na dysku twardym i będzie łatwiejsze do przenoszenia. Dodatkowo kompilatora nie trzeba instalować, dzięki czemu cały system będzie szybki w użyciu. Środowisko kompilacji wymaga systemu operacyjnego Windows. Środowisko to znajduje się w załączniku razem z programem.


\section{Tworzenie szablonów raportów do systemu rekrutacji}

W rekrutacji na uczelnie wykorzystuje się dokumenty, które należało dokładnie odwzorować w nowym systemie. Są to następujące dokumenty:\\
\begin{enumerate}
\item protokół przekazania
\item listy potwierdzenia podjęcia studiów 
\item listy rankingowe 
\item listy przyjętych
\item listy nieprzyjętych
\item decyzja o przyjęciu danego kandydata
\item decyzja o nieprzyjęciu danego kandydata 
\end{enumerate}
\vspace{5mm}
Przy tworzeniu szablonów wystąpiły problemy, które należało rozwiązać.  W dużej mierze problemy te powtarzają się,  opisane więc zostały tylko rozwiązania tych problemów a nie każdy szablon raportu. W poniższych podsekcjach znajdują się przedstawione problemy oraz ich rozwiązania.
\subsection{Definiowanie zmiennych}



\subsection{Wyświetlanie listy}
Problem występujący w protokole przekazania oraz we wszystkich listach. Przykładowo potrzebujemy wyświetlić poniższą listę, gdzie oczywiście wartości pochodzą z bazy danych:
\begin{lstlisting}
Lp. Tok studiów Liczba kopert
1 Informatyka — niestacjonarne STUDIA pierwszego stopnia 1455
2 Informatyka — stacjonarne STUDIA pierwszego stopnia 729
3 Mechatronika — niestacjonarne STUDIA pierwszego stopnia 1447
...
\end{lstlisting}

\subsection{Grupowanie}
\subsection{Aktualizacja daty}
\chapter{Testowanie Systemu}

Przed wdrożeniem programu do realnego systemu, program należy przetestować. Testy powinny być prowadzone na tymczasowej bazie danych, ponieważ idea testów jest taka, że po podmianie bazy danych na realną wszystko ma działać bez zmian. Zmienić się może tylko zapis połączenia z bazą danych w pliku konfiguracyjnym. Dzięki takiemu zabiegowi, będzie można być pewnym tego, że wszystko będzie działać na prawdziwej bazie danych.  Do przeprowadzenia testów potrzebne będzie  odtworzyć przyszłe środowisko, w którym będzie działać program, przygotować szablony dokumentów, które są wytwarzane w procesie rekrutacji oraz wygenerować dokumenty. Ostatnim już krokiem będzie sprawdzenie czy podczas tego procesu nie ma żadnych komplikacji oraz czy wygenerowane dokumenty nie zawierają błędów.

\section{ Generowanie przykładowych danych}

W celu przetestowania systemu generowania raportów w procesie rekrutacji kandydatów na studia potrzebne będą testowe dane w dokładnie tej samej strukturze co w systemie rekrutacji, ponieważ w szablonach latexu znajdują się zapytania SQL do danych tabel w bazie danych. 
W celu otrzymania tych danych potrzebne będzie:
1.	Utworzenie nowej bazy danych na silniku Firebird’a 
2.	Stworzenie wystarczającej struktury tabel odzwierciedlającą strukturę w systemie uczelnianym.
3.	Wygenerowanie dużej ilości testowych danych osobowych.
4.	Uzupełnienie tabel danymi, które zostały wygenerowane wcześniej oraz dodanie do nich dodatkowych, jednocześnie losowych, informacji na temat procesu rekrutacji.
Po wykonaniu tych kroków, powinna powstać baza danych do której bez problem program połączy się i wyciągnie z niej potrzebne dane dokładnie jak z realnej bazy danych.

\subsection{ Tworzenie bazy danych }

Do stworzenia pliku bazy danych na silniku firebird’a posłużyć się można narzędziem dostepnym w katalogu bin zainstalowanego serwera firebird.  Narzędzie to pozwala z linii komend tworzyć i łączyć się z bazami danych. W tym przypadku użyta zostanie komenda „CREATE DATABASE”
\begin{lstlisting}
C:\Program Files\Firebird\bin>isql
SQL>CREATE DATABASE 'D:\test systemu\rekrutacja.fdb'
CON>user 'SYSDBA' password 'masterkey';
\end{lstlisting}
Po wykonaniu tego polecenia zostanie utworzona baza danych. Takie same dane należy teraz wpisać do pliku konfiguracyjnego programu czyli DBRaportLatex.bat aby program mogł się połączyć z tą bazą:
\begin{lstlisting}
dbengine=firebirdsql
hostname=//localhost
port=3050
dbpath=D:\test systemu\rekrutacja.fdb
user=SYSDBA
password= masterkey
\end{lstlisting}
\subsection{ Stworzenie struktury}

Dotychczasowy system wykorzystywał tabelę (widok) która była generowana dynamicznie i która zawiera wszystkich studentów w rekrutacji. Zawiera ona wszystkie dane potrzebne do wytworzenia dokumentów. Jeden rekord to jeden student ze wszystkimi informacjami na jego temat. Dodatkowo jeszcze potrzebna jest tabela z informacjami na temat rekrutacji, takimi jak na przykład nazwisko i imię przewodniczącego komisji, czy data wydania decyzji przyjęcia studenta. Z tych tabel będą pobierane informacje, natomiast do wygenerowania danych potrzebne będą dwie dodatkowe tymczasowe tabele.
Tabela główna z kandydatami(zapis SQL):
\begin{lstlisting}
CREATE TABLE KANDYDAT_ALIGEZA (
    STUD_ID                        INTEGER NOT NULL,
    OSOBA_ID                       INTEGER NOT NULL,
    STUD_NRTECZKI                  INTEGER,
    NAZWISKO                       VARCHAR(100),
    IMIE                           VARCHAR(100),
    NAZWISKOIMIONA                 VARCHAR(200),
    ADR_ULICA_MIEJSCOWOSC_NR_DOMU  VARCHAR(200),
    ADR_KOD_POCZTOWY_POCZTA        VARCHAR(100),
    STUDIA_NAZWA                   VARCHAR(100),
    STUD_ILPUNKTOW                 INTEGER,
    STUD_ILPUNKTOWKREM             INTEGER,
    TOKNAUKI_NAZWATOKU             VARCHAR(200),
    KIERUNEK                       VARCHAR(100),
    SPEC_ID                        INTEGER,
    DATAPRZYJECIAPODANIA           DATE,
    TOKNAUKI_ID                    INTEGER,
    OSOBA_PESEL                    VARCHAR(50),
    KIERUNEK_MY                    VARCHAR(200),
    FORMA_STUDIOW_MY               VARCHAR(200),
    STOPIEN_STUDIOW_MY             VARCHAR(100),
    KIERUNEK_FORMA_SKROT_MY        VARCHAR(100),
    NR_DECYZJI                     VARCHAR(100),
    CZY_PRZYJETY                   INTEGER,
    DATA_DECYZJI                   DATE,
    ILE_PUNKTOW                    INTEGER,
    PANPANI                        CHAR(1)
);
\end{lstlisting}
Tabela z dodatkowymi informacjami(wartości przypisane są do kluczy tekstowych, jest to tablica asocjacyjna):
\begin{lstlisting}
CREATE TABLE SETUP_ALIGEZA (
    KLUCZ    VARCHAR(50) NOT NULL,
    WARTOSC  VARCHAR(100)
);
\end{lstlisting}
Tymczasowa tabela do zaimportowania listy imion i nazwisk oraz losowych peseli.
\begin{lstlisting}
CREATE TABLE DANE (
    IMIE_NAZ  VARCHAR(200),
    ADRES     VARCHAR(200),
    PESEL     VARCHAR(50),
    NAZWISKO  VARCHAR(100),
    IMIE      VARCHAR(100)
);
\end{lstlisting}
Tymczasowa tabela do procedury losowego uzupełniania informacji o rekrucie o kierunku jaki wybrał.
\begin{lstlisting}
CREATE TABLE TOKNAUKI_ALIGEZA (
    TOKNAUKI_ID              INTEGER NOT NULL,
    KIERUNEK_MY              VARCHAR(50),
    FORMA_STUDIOW_MY         VARCHAR(50),
    STOPIEN_STUDIOW_MY       VARCHAR(50),
    KIERUNEK_FORMA_SKROT_MY  VARCHAR(10),
    LICZBA_MIEJSC            SMALLINT,
    DATA_DECYZJI_OD          TIMESTAMP,
    DATA_DECYZJI_DO          TIMESTAMP,
    KOD_IKR                  VARCHAR(3)
);
\end{lstlisting}
\subsection{ Generowanie testowych danych osobowych}

Do wygenerowania kandydatów potrzeba imienia nazwiska oraz adresu. Takie dane dostępne są w książkach telefonicznych. Posługując się jedną z takich książek stworzony został plik csv o separatorze „;” zawierający po kolei imię z nazwiskiem, adres, pesel, nazwisko oraz imie. Pesel został dodany do każdej osoby jako losowy ciąg cyfr spełniający walidację peselu. Ze względu na fakt iż pesel został wygenerowany losowo, może zdarzyć się iż mężczyzna posiadać będzie kobiecy pesel, w następstwie czego, we wygenerowanych dokumentach wypisane zostanie „Pani” i na odwrót.
Struktura pliku:
\begin{lstlisting}
 Abram Andrzej; Lwowska 116;88071640299;Abram;Andrzej
 Abram Halina; Ludwika Zamenhofa 2;86111210691;Abram;Halina
...

\end{lstlisting}

Tak sformatowany plik CSV,  łatwo zaimportować do bazy danych do tabeli „dane” ze względu na identyczną kolejność danych w kolumnach. Do importu wykorzystana została funkcja programu IBExpert „import data”. Jedna linijka w pliku zostaje zaimportowana jako jeden rekord, w którym każde pole po kolei odpowiada wartościom między średnikami. Zaimportowanych w ten sposób zostało 10001 rekordów (osób) do tabeli „dane” do dalszych manipulacji.

\subsection{Generowanie kandydatów na studentów}

Kolejnym krokiem jest uzupełnienie tabeli z tokami studiów. W testach dodanych zostało 8 przykładowych toków nauki. 

\begin{lstlisting}
1 Informatyka	niestacjonarne	pierwszego stopnia	INF-n				
2 Mechatronika	niestacjonarne	pierwszego stopnia	MT-n				
3 Mechatronika	stacjonarne	pierwszego stopnia	MT-s				
...


\end{lstlisting}
Uzupełnienia wymaga także tabela z dodatkowymi informacjami „SETUP\_ALIGEZA” przykładowymi danymi:
\begin{lstlisting}
dataWydaniaDecyzji	09.10.2015
miejsceWydaniaDecyzji	Nowy Sącz
przewodniczacyIKR	mgr inż. Sławomir Jurkowski
rokAkademicki	2015/2016
czyUwzglednicDateWydaniaDecyzji	N
...

\end{lstlisting}

Mając już to wszystko potrzebna jest procedura, która utworzy listę kandydatów z tych wszystkich danych. 

\begin{lstlisting}
create procedure GENERUJ
returns (
    TESTCHAR varchar(50),
    TEST integer)
as
declare variable IMIE varchar(100);
declare variable NAZ varchar(100);
declare variable IMIENAZ varchar(200);
declare variable ADRES varchar(200);
declare variable PESEL varchar(50);
declare variable LICZNIK integer;
declare variable STOPIEN varchar(50);
declare variable KIERUNEK varchar(50);
declare variable FORMA varchar(50);
declare variable SKROT varchar(10);
declare variable RANDINT integer;
declare variable PUNKTY integer;
declare variable CZY_PRZYJETY integer;
declare variable DATA_DEC varchar(100);
begin
licznik = 1;
for select * from dane into
:imienaz,:adres,:pesel,:naz,:imie
do
begin
randint = CAST(round(rand()*7+1) as INTEGER);
punkty = CAST(round(rand()*500) as INTEGER);
if(punkty > 250) then czy_przyjety = 1;
if(punkty <= 250) then czy_przyjety = 2;

select kierunek_my,forma_studiow_my,stopien_studiow_my,
kierunek_forma_skrot_my
FROM toknauki_aligeza where toknauki_id = :randint
INTO :kierunek,:forma,:stopien,:skrot;

select wartosc FROM setup_aligeza 
WHERE klucz='dataWydaniaDecyzji'
INTO :data_dec;

INSERT INTO kandydat_aligeza
(stud_id,osoba_id,stud_nrteczki,nazwisko,imie,
nazwiskoimiona,adr_ulica_miejscowosc_nr_domu,
adr_kod_pocztowy_poczta,osoba_pesel,panpani,
studia_nazwa,toknauki_nazwatoku,kierunek,
kierunek_my,forma_studiow_my,stopien_studiow_my,
kierunek_forma_skrot_my,    stud_ilpunktow,
stud_ilpunktowkrem,ile_punktow,czy_przyjety,
data_decyzji,nr_decyzji,dataprzyjeciapodania)
VALUES (:licznik,:licznik,cast(round(rand()*200+1) as integer),
:naz,:imie,:imienaz,:adres, 
cast( 'Nowy Sacz 33-300' as varchar(100)),
:pesel,cast( 'M' as char(1)),    :forma,:kierunek || 
' N inz. 3.50 2015/2016 zimowy',:kierunek,:kierunek,:forma,
:stopien,:skrot,    :punkty,:punkty,:punkty,:czy_przyjety,
cast(:data_dec as DATE),'328/2015','2015-08-14');

licznik = :licznik + 1;
end
test = :licznik;
suspend;
end
\end{lstlisting}
Powyższa procedura z jednej osoby z tabeli dane tworzy jednego kandydata, losując mu tok nauki, ilość punktów oraz czy zostanie przyjęty lub nie. Dorzucane są także pewne stałe wartości, podobne do tych w oryginalnej bazie danych, które nie wymagają uzmiennienia. Procedura ta, po jednorazowym wywołaniu, wygenerowała 10001 rekordów w tabeli "kandydat\_aligeza". Daje to wystarczającą ilość testowych kandydatów do przeprowadzenia testów.

\section{ Przygotowanie szablonów raportów}



\appendix

% tutaj załączniki

%\chapter*{Bibliografia}
\nocite{*}
\bibliographystyle{plplain}
%\bibliographystylebk{plplain}
%\bibliographystylest{plplain}
%\bibliographystyledoc{plplain}
% \bibliographystyleweb{plplain}
%\bibliographybk{BIB/books}
%\bibliographyst{BIB/books}
%\bibliographydoc{BIB/books}
% \bibliographyweb{BIB/books}

% \bibliography{bib/verificard,bib/jml,bib/daikon}
\bibliography{bib/daikon,bib/statistics,bib/other}

\end{document}

% ex: set tabstop=4 shiftwidth=4 softtabstop=4 noexpandtab fileformat=unix filetype=tex spelllang=pl,en spell:

